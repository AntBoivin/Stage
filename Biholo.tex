
\newif\ifwhole

\wholetrue
% Ajouter \wholetrue si on compile seulement ce fichier

\ifwhole
 \documentclass[a4page,10pt]{article}
     \usepackage[Latin1]{inputenc}
\usepackage[francais]{babel}
\usepackage{amsmath,amssymb,amsthm}
\usepackage{textcomp}
\usepackage{mathrsfs}
\usepackage{algcompatible,algorithm  }
\usepackage[all]{xy}
\usepackage{hyperref}
\usepackage{fancyhdr}
\usepackage{supertabular}
\pagestyle{plain}

\toks0=\expandafter{\xy}
\edef\xy{\noexpand\shorthandoff{!?;:}\the\toks0 }
\makeatletter

\renewcommand*{\ALG@name}{Algorithme}

\makeatother

\renewcommand{\algorithmicrequire}{\textbf{\textsc {Entr\'ees  :  } } }
\renewcommand{\algorithmicensure}{\textbf{\textsc { Sortie  :  } } }
\renewcommand{\algorithmicwhile}{\textbf{Tant que}}
\renewcommand{\algorithmicdo}{\textbf{faire }}
\renewcommand{\algorithmicif}{\textbf{Si}}
\renewcommand{\algorithmicelse}{\textbf{Sinon}}
\renewcommand{\algorithmicthen}{\textbf{alors }}
\renewcommand{\algorithmicend}{\textbf{fin}}
\renewcommand{\algorithmicfor}{\textbf{Pour}}
\renewcommand{\algorithmicuntil}{\textbf{Jusqu'\`a}}
\renewcommand{\algorithmicrepeat}{\textbf{Répéter}}

\newcommand{\A}{\mathbb{A}}
\newcommand{\B}{\mathbb{B}}
\newcommand{\C}{\mathbb{C}}
\newcommand{\D}{\mathbb{D}}
\newcommand{\E}{\mathbb{E}}
\newcommand{\F}{\mathbb{F}}
\newcommand{\G}{\mathbb{G}}
\renewcommand{\H}{\mathbb{H}}
\newcommand{\I}{\mathbb{I}}
\newcommand{\J}{\mathbb{J}}
\newcommand{\K}{\mathbb{K}}
\renewcommand{\L}{\mathbb{L}}
\newcommand{\M}{\mathbb{M}}
\newcommand{\N}{\mathbb{N}}
\renewcommand{\O}{\mathbb{O}}
\renewcommand{\P}{\mathbb{P}}
\newcommand{\Q}{\mathbb{Q}}
\newcommand{\R}{\mathbb{R}}
\renewcommand{\S}{\mathbb{S}}
\newcommand{\T}{\mathbb{T}}
\newcommand{\U}{\mathbb{U}}
\newcommand{\V}{\mathbb{V}}
\newcommand{\W}{\mathbb{W}}
\newcommand{\X}{\mathbb{X}}
\newcommand{\Y}{\mathbb{Y}}
\newcommand{\Z}{\mathbb{Z}}
\newcommand{\Acal}{\mathcal{A}}
\newcommand{\Bcal}{\mathcal{B}}
\newcommand{\Ccal}{\mathcal{C}}
\newcommand{\Dcal}{\mathcal{D}}
\newcommand{\Ecal}{\mathcal{E}}
\newcommand{\Fcal}{\mathcal{F}}
\newcommand{\Gcal}{\mathcal{G}}
\newcommand{\Hcal}{\mathcal{H}}
\newcommand{\Ical}{\mathcal{I}}
\newcommand{\Jcal}{\mathcal{J}}
\newcommand{\Kcal}{\mathcal{K}}
\newcommand{\Lcal}{\mathcal{L}}
\newcommand{\Mcal}{\mathcal{M}}
\newcommand{\Ncal}{\mathcal{N}}
\newcommand{\Ocal}{\mathcal{O}}
\newcommand{\Pcal}{\mathcal{P}}
\newcommand{\Qcal}{\mathcal{Q}}
\newcommand{\Rcal}{\mathcal{R}}
\newcommand{\Scal}{\mathcal{S}}
\newcommand{\Tcal}{\mathcal{T}}
\newcommand{\Ucal}{\mathcal{U}}
\newcommand{\Vcal}{\mathcal{V}}
\newcommand{\Wcal}{\mathcal{W}}
\newcommand{\Xcal}{\mathcal{X}}
\newcommand{\Ycal}{\mathcal{Y}}
\newcommand{\Zcal}{\mathcal{Z}}
\newcommand{\Ascr}{\mathscr{A}}
\newcommand{\Bscr}{\mathscr{B}}
\newcommand{\Cscr}{\mathscr{C}}
\newcommand{\Dscr}{\mathscr{D}}
\newcommand{\Escr}{\mathscr{E}}
\newcommand{\Fscr}{\mathscr{F}}
\newcommand{\Gscr}{\mathscr{G}}
\newcommand{\Hscr}{\mathscr{H}}
\newcommand{\Iscr}{\mathscr{I}}
\newcommand{\Jscr}{\mathscr{J}}
\newcommand{\Kscr}{\mathscr{K}}
\newcommand{\Lscr}{\mathscr{L}}
\newcommand{\Mscr}{\mathscr{M}}
\newcommand{\Nscr}{\mathscr{N}}
\newcommand{\Oscr}{\mathscr{O}}
\newcommand{\Pscr}{\mathscr{P}}
\newcommand{\Qscr}{\mathscr{Q}}
\newcommand{\Rscr}{\mathscr{R}}
\newcommand{\Sscr}{\mathscr{S}}
\newcommand{\Tscr}{\mathscr{T}}
\newcommand{\Uscr}{\mathscr{U}}
\newcommand{\Vscr}{\mathscr{V}}
\newcommand{\Wscr}{\mathscr{W}}
\newcommand{\Xscr}{\mathscr{X}}
\newcommand{\Yscr}{\mathscr{Y}}
\newcommand{\Zscr}{\mathscr{Z}}
\newcommand{\Afrak}{\mathfrak{A}}
\newcommand{\Bfrak}{\mathfrak{B}}
\newcommand{\Cfrak}{\mathfrak{C}}
\newcommand{\Dfrak}{\mathfrak{D}}
\newcommand{\Efrak}{\mathfrak{E}}
\newcommand{\Ffrak}{\mathfrak{F}}
\newcommand{\Gfrak}{\mathfrak{G}}
\newcommand{\Hfrak}{\mathfrak{H}}
\newcommand{\Ifrak}{\mathfrak{I}}
\newcommand{\Jfrak}{\mathfrak{J}}
\newcommand{\Kfrak}{\mathfrak{K}}
\newcommand{\Lfrak}{\mathfrak{L}}
\newcommand{\Mfrak}{\mathfrak{M}}
\newcommand{\Nfrak}{\mathfrak{N}}
\newcommand{\Ofrak}{\mathfrak{O}}
\newcommand{\Pfrak}{\mathfrak{P}}
\newcommand{\Qfrak}{\mathfrak{Q}}
\newcommand{\Rfrak}{\mathfrak{R}}
\newcommand{\Sfrak}{\mathfrak{S}}
\newcommand{\Tfrak}{\mathfrak{T}}
\newcommand{\Ufrak}{\mathfrak{U}}
\newcommand{\Vfrak}{\mathfrak{V}}
\newcommand{\Wfrak}{\mathfrak{W}}
\newcommand{\Xfrak}{\mathfrak{X}}
\newcommand{\Yfrak}{\mathfrak{Y}}
\newcommand{\Zfrak}{\mathfrak{Z}}
\newcommand{\afrak}{\mathfrak{a}}
\newcommand{\bfrak}{\mathfrak{b}}
\newcommand{\cfrak}{\mathfrak{c}}
\newcommand{\dfrak}{\mathfrak{d}}
\newcommand{\efrak}{\mathfrak{e}}
\newcommand{\ffrak}{\mathfrak{f}}
\newcommand{\gfrak}{\mathfrak{g}}
\newcommand{\hfrak}{\mathfrak{h}}
\newcommand{\ifrak}{\mathfrak{i}}
\newcommand{\jfrak}{\mathfrak{j}}
\newcommand{\kfrak}{\mathfrak{k}}
\newcommand{\lfrak}{\mathfrak{l}}
\newcommand{\mfrak}{\mathfrak{m}}
\newcommand{\nfrak}{\mathfrak{n}}
\newcommand{\ofrak}{\mathfrak{o}}
\newcommand{\pfrak}{\mathfrak{p}}
\newcommand{\qfrak}{\mathfrak{q}}
\newcommand{\rfrak}{\mathfrak{r}}
\newcommand{\sfrak}{\mathfrak{s}}
\newcommand{\tfrak}{\mathfrak{t}}
\newcommand{\ufrak}{\mathfrak{u}}
\newcommand{\vfrak}{\mathfrak{v}}
\newcommand{\wfrak}{\mathfrak{w}}
\newcommand{\xfrak}{\mathfrak{x}}
\newcommand{\yfrak}{\mathfrak{y}}
\newcommand{\zfrak}{\mathfrak{z}}
\newcommand{\Abar}{\overline{A}}
\newcommand{\Bbar}{\overline{B}}
\newcommand{\Cbar}{\overline{C}}
\newcommand{\Dbar}{\overline{D}}
\newcommand{\Ebar}{\overline{E}}
\newcommand{\Fbar}{\overline{F}}
\newcommand{\Gbar}{\overline{G}}
\newcommand{\Hbar}{\overline{H}}
\newcommand{\Ibar}{\overline{I}}
\newcommand{\Jbar}{\overline{J}}
\newcommand{\Kbar}{\overline{K}}
\newcommand{\Lbar}{\overline{L}}
\newcommand{\Mbar}{\overline{M}}
\newcommand{\Nbar}{\overline{N}}
\newcommand{\Obar}{\overline{O}}
\newcommand{\Pbar}{\overline{P}}
\newcommand{\Qbar}{\overline{Q}}
\newcommand{\Rbar}{\overline{R}}
\newcommand{\Sbar}{\overline{S}}
\newcommand{\Tbar}{\overline{T}}
\newcommand{\Ubar}{\overline{U}}
\newcommand{\Vbar}{\overline{V}}
\newcommand{\Wbar}{\overline{W}}
\newcommand{\Xbar}{\overline{X}}
\newcommand{\Ybar}{\overline{Y}}
\newcommand{\Zbar}{\overline{Z}}
\newcommand{\abar}{\overline{a}}
\newcommand{\bbar}{\overline{b}}
\newcommand{\cbar}{\overline{c}}
\newcommand{\dbar}{\overline{d}}
\newcommand{\ebar}{\overline{e}}
\newcommand{\fbar}{\overline{f}}
\newcommand{\gbar}{\overline{g}}
\renewcommand{\hbar}{\overline{h}}
\newcommand{\ibar}{\overline{i}}
\newcommand{\jbar}{\overline{j}}
\newcommand{\kbar}{\overline{k}}
\newcommand{\lbar}{\overline{l}}
\newcommand{\mbar}{\overline{m}}
\newcommand{\nbar}{\overline{n}}
\newcommand{\obar}{\overline{o}}
\newcommand{\pbar}{\overline{p}}
\newcommand{\qbar}{\overline{q}}
\newcommand{\rbar}{\overline{r}}
\newcommand{\sbar}{\overline{s}}
\newcommand{\tbar}{\overline{t}}
\newcommand{\ubar}{\overline{u}}
\newcommand{\vbar}{\overline{v}}
\newcommand{\wbar}{\overline{w}}
\newcommand{\xbar}{\overline{x}}
\newcommand{\ybar}{\overline{y}}
\newcommand{\zbar}{\overline{z}}
\newcommand\bigzero{\makebox(0,0){\text{\huge0}}}
\newcommand{\limp}{\lim\limits_{\leftarrow}}
\newcommand{\limi}{\lim\limits_{\rightarrow}}


\DeclareMathOperator{\End}{\mathrm{End}}
\DeclareMathOperator{\Hom}{\mathrm{Hom}}
\DeclareMathOperator{\Vect}{\mathrm{Vect}}
\DeclareMathOperator{\Spec}{\mathrm{Spec}}
\DeclareMathOperator{\multideg}{\mathrm{multideg}}
\DeclareMathOperator{\LM}{\mathrm{LM}}
\DeclareMathOperator{\LT}{\mathrm{LT}}
\DeclareMathOperator{\LC}{\mathrm{LC}}
\DeclareMathOperator{\PPCM}{\mathrm{PPCM}}
\DeclareMathOperator{\PGCD}{\mathrm{PGCD}}
\DeclareMathOperator{\Syl}{\mathrm{Syl}}
\DeclareMathOperator{\Res}{\mathrm{Res}}
\DeclareMathOperator{\Com}{\mathrm{Com}}
\DeclareMathOperator{\GL}{\mathrm{GL}}
\DeclareMathOperator{\SL}{\mathrm{SL}}
\DeclareMathOperator{\SU}{\mathrm{SU}}
\DeclareMathOperator{\PGL}{\mathrm{PGL}}
\DeclareMathOperator{\PSL}{\mathrm{PSL}}
\DeclareMathOperator{\PSU}{\mathrm{PSU}}
\DeclareMathOperator{\SO}{\mathrm{SO}}
\DeclareMathOperator{\Sp}{\mathrm{Sp}}
\DeclareMathOperator{\Spin}{\mathrm{Spin}}
\DeclareMathOperator{\Ker}{\mathrm{Ker}}
%\DeclareMathOperator{\Im}{\mathrm{Im}}

\DeclareMathOperator{\Ens}{\mathbf{Ens}}
\DeclareMathOperator{\Top}{\mathbf{Top}}
\DeclareMathOperator{\Ann}{\mathbf{Ann}}
\DeclareMathOperator{\Gr}{\mathbf{Gr}}
\DeclareMathOperator{\Ab}{\mathbf{Ab}}
%\DeclareMathOperator{\Vect}{\mathbf{Vect}}
\DeclareMathOperator{\Mod}{\mathbf{Mod}}

\newcommand{\bishop}{bishop_conditions_1964}
\newcommand{\cac}{barlet_cycles_2014}
\newcommand{\btg}{bourbaki_topologie_2007}   
\newcommand{\pag}{griffiths_principles_2011}
\newcommand{\barlet}{barlet_espace_1975}
\newcommand{\voisin}{voisin_theorie_2002}
\newcommand{\scv}{grauert_several_2013}
\newcommand{\bredon}{bredon_topology_1993}
\newcommand{\BM}{bochner_groups_1947}
\newcommand{\Lee}{lee_structure_2001}
\newcommand{\laf}{lafontaine_introduction_2012}
\newcommand{\ccs}{barth_compact_2004}
\newcommand{\cas}{grauert_coherent_1984}
\newcommand{\akh}{akhiezer_lie_2012}
\newcommand{\catan}{catanese_moduli_1988}
\newcommand{\Hart}{hartshorne_algebraic_1977}
\newcommand{\Hatcher}{hatcher_algebraic_2002}

%\headheight=0mm
%\topmargin=-20mm
\oddsidemargin=31pt
\evensidemargin=31pt
%\textwidth=18cm
%\textheight=25cm
%\setlength{\marginparwidth}{0pt}

\parindent=0mm
\newif\ifproof
\newcommand{\demo}[1]{\ifproof #1 \else \fi}
 %Instruction d'utilisation : 
%les preuves du texte sont, en principe, entre des balises \demo, en sus des \begin{proof} pour l'instant.
%Laisser le texte tel quel, fait qu'elles ne sont pas affich�es.
%Mettre \prooftrue fait que toutes les preuves jusqu'� un \prooffalse ou la fin du document. 


 \begin{document}
\newtheorem{Thm}{Th�or�me}[chapter]
\newtheorem{Prop}[Thm]{Proposition}
\newtheorem{Propte}[Thm]{Propri�t�}
\newtheorem{Lemme}[Thm]{Lemme}
\newtheorem{Cor}[Thm]{Corollaire}


\theoremstyle{definition}

\newtheorem{Ex}[Thm]{Exemple}
\newtheorem{Def}[Thm]{D�finition}
\newtheorem{Defpropte}[Thm]{D�finition et propri�t�}
\newtheorem{Defprop}[Thm]{D�finition et proposition}
\newtheorem{Defthm}[Thm]{Th�or�me et d�finition}
\newtheorem{Not}[Thm]{Notation}
\newtheorem{Conv}[Thm]{Convention}
\newtheorem{Const}[Thm]{Construction}

\theoremstyle{remark}
\newtheorem{Rq}[Thm]{Remarque}
\newtheorem{Slog}[Thm]{Slogan}
\newtheorem{Exo}[Thm]{Exercice}

\fi

\begin{Thm}\label{biholo}
Soit $X$ un espace analytique complexe compact et $f$ un automorphisme
de $X$. Soit $\Cscr_\Gamma$ la composante irr�ductible de l'espace des
cycles contenant le graphe de $f$. Alors, l'ensemble des points de
$\Cscr_\Gamma$ correspondant aux automorphismes de $X$ est un ouvert de
Zariski.
\end{Thm}
\begin{proof}
  On supposera $X$ lisse et irr�ductible. \\
Soit $Z \subset X \times X \times \Cscr_\Gamma$ le cycle universel.  La
projection naturelle $\pi_3 : Z \to \Cscr_\Gamma$ est propre (cf
\cite{\barlet} th�or�me 1).\\
Quitte � remplacer $\Cscr_\Gamma$ par un ouvert de Zariski, on peut
supposer $\Cscr_\Gamma$ et $Z$ lisse. En effet, les lieux singuliers de
$\Cscr_\Gamma$ et $Z$ sont des espaces analytiques. L'image par $\pi_3$
de ce dernier est aussi un espace analytique et donc son compl�mentaire
est un ouvert de Zariski. L'image r�ciproque de cet ouvert est donc lisse.
Par la proposition 1.21 p 108 de \cite{\scv}, il existe un ouvert de
Zariski $U$ tel que $\pi_{3\mid U}$ soit une submersion. Par les m�mes
consid�rations que pour la lissit� de $Z$, $\pi_3$ est lisse sur un
ouvert de Zariski de $\Cscr_\Gamma$. Ainsi, toutes les fibres sur cet
ouvert sont lisses. \\



De la m�me fa�on, pour les projections $\pi_i : Z \to X$, on obtient
que, sur un ouvert de Zariski $S_i$ de $Z$, ce sont des submersions.
Elles induisent donc pour tout $c \in \Cscr_\Gamma$, une submersion
$\pi_{i,c} : U_c \subset Z_c\to X$ car pour tout $y \in X$, $\pi_i^{-1}(y)$ et $Z_c$ s'intersectent transversalement (car pour tout $x \in \pi_i^{-1}(y) $, $T_x\pi^{-1}(y)\subset Kerd_x\pi_i$) et donc $\pi_{i,c}^{-1}(y)=\pi^{-1}(y) \cap Z_c$ est une vari�t� lisse de codimension
\[codim(\pi_i^{-1}(y))+\codim(Z_c)=dim(Z)-dim(\Cscr_\Gamma)+dim(Z)-dim(X)=dim(Z)\]
i.e de dimension nulle. \\
On en d�duit, pour des questions de dimension, que $d\pi_i$ est un isomorphisme. L'application $\pi_i$ est une surjection
car si ce n'�tait pas le cas $\pi_{i,c}(U_c)$ serait un espace
analytique de dimension strictement inf�rieur � $n$. Or, cela voudrait
que ses fibres serait de dimension sup�rieure ou �gale � $1$. Or, ce
n'est pas le cas car c'est un diff�omorphisme local. \\
Pour finir, montrons que les points de $\Cscr_\Gamma \setminus\pi_3(S_1^c \cup S_2^c)$ correspondent aux automorphismes de $X$. \\
Soit $c$ un cycle de $\Cscr_\Gamma$. \\
Les applications $\pi_i : Z_C \to X$ est un rev�tement propre par
\ref{revet}. \\
Comme $\Cscr_\Gamma$ est connexe et localement connexe par arcs alors il
est connexe par arcs. Il existe donc un chemin continu $\gamma : [0,1] \to
\Cscr_\Gamma$ entre $c$ et $\Gamma$. \\
Soit $x \in X$. Il existe un ouvert de $X$ autour de $x$ qui est biholomorphe � un polydisque ouvert relativement compact $B$ (carte holomorphe). Ainsi, pour $U$ un polydisque inclus dans $f(B)$, $(B,U,\pi_i)$ d�finit une �caille adapt�e au cycle $\Gamma$. Par continuit� de la famille $(\gamma(t))_{t \in [0,1]})$ et la proposition \ref{famille_cont}, le degr� du rev�tement reste constant, ce qui permet de conclure que $pi_{i,c}$ a le m�me nombre de feuilles que $\pi_{i,\Gamma}$ i.e. 1. \\
$\pi_i$ est donc un hom�omorphisme et donc par l'isomorphisme pr�c�dent et le
th�or�me d'inversion locale, c'est un biholomorphisme.

\end{proof}

\begin{Prop}\label{famille_cont}
  Soit $(Z_s)_{s \in S}$ une famille de $n$-cycles d'un espace analytique $X$. Alors cette famille est continue en $s_0 \in S$ si, et seulement si, pour toute �caille $E=(U,B,f)$ adapt�e � $X_{s_0}$, il existe un ouvert $S_0$ de $s_0$ dans $S$, tel que, pour tout $s \in S_0$, l'�caille $E$ soit adapt�e � $X_s$ et $\deg_E(X_s)=\deg_E(X_{s_0})$.



\end{Prop}




\ifwhole
 \end{document}
\fi