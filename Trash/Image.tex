
\newif\ifwhole

\wholetrue
% Ajouter \wholetrue si on compile seulement ce fichier

\ifwhole
 \documentclass[a4page,10pt]{article}
     \usepackage[Latin1]{inputenc}
\usepackage[francais]{babel}
\usepackage{amsmath,amssymb,amsthm}
\usepackage{textcomp}
\usepackage{mathrsfs}
\usepackage{algcompatible,algorithm  }
\usepackage[all]{xy}
\usepackage{hyperref}
\usepackage{fancyhdr}
\usepackage{supertabular}
\pagestyle{plain}

\toks0=\expandafter{\xy}
\edef\xy{\noexpand\shorthandoff{!?;:}\the\toks0 }
\makeatletter

\renewcommand*{\ALG@name}{Algorithme}

\makeatother

\renewcommand{\algorithmicrequire}{\textbf{\textsc {Entr\'ees  :  } } }
\renewcommand{\algorithmicensure}{\textbf{\textsc { Sortie  :  } } }
\renewcommand{\algorithmicwhile}{\textbf{Tant que}}
\renewcommand{\algorithmicdo}{\textbf{faire }}
\renewcommand{\algorithmicif}{\textbf{Si}}
\renewcommand{\algorithmicelse}{\textbf{Sinon}}
\renewcommand{\algorithmicthen}{\textbf{alors }}
\renewcommand{\algorithmicend}{\textbf{fin}}
\renewcommand{\algorithmicfor}{\textbf{Pour}}
\renewcommand{\algorithmicuntil}{\textbf{Jusqu'\`a}}
\renewcommand{\algorithmicrepeat}{\textbf{Répéter}}

\newcommand{\A}{\mathbb{A}}
\newcommand{\B}{\mathbb{B}}
\newcommand{\C}{\mathbb{C}}
\newcommand{\D}{\mathbb{D}}
\newcommand{\E}{\mathbb{E}}
\newcommand{\F}{\mathbb{F}}
\newcommand{\G}{\mathbb{G}}
\renewcommand{\H}{\mathbb{H}}
\newcommand{\I}{\mathbb{I}}
\newcommand{\J}{\mathbb{J}}
\newcommand{\K}{\mathbb{K}}
\renewcommand{\L}{\mathbb{L}}
\newcommand{\M}{\mathbb{M}}
\newcommand{\N}{\mathbb{N}}
\renewcommand{\O}{\mathbb{O}}
\renewcommand{\P}{\mathbb{P}}
\newcommand{\Q}{\mathbb{Q}}
\newcommand{\R}{\mathbb{R}}
\renewcommand{\S}{\mathbb{S}}
\newcommand{\T}{\mathbb{T}}
\newcommand{\U}{\mathbb{U}}
\newcommand{\V}{\mathbb{V}}
\newcommand{\W}{\mathbb{W}}
\newcommand{\X}{\mathbb{X}}
\newcommand{\Y}{\mathbb{Y}}
\newcommand{\Z}{\mathbb{Z}}
\newcommand{\Acal}{\mathcal{A}}
\newcommand{\Bcal}{\mathcal{B}}
\newcommand{\Ccal}{\mathcal{C}}
\newcommand{\Dcal}{\mathcal{D}}
\newcommand{\Ecal}{\mathcal{E}}
\newcommand{\Fcal}{\mathcal{F}}
\newcommand{\Gcal}{\mathcal{G}}
\newcommand{\Hcal}{\mathcal{H}}
\newcommand{\Ical}{\mathcal{I}}
\newcommand{\Jcal}{\mathcal{J}}
\newcommand{\Kcal}{\mathcal{K}}
\newcommand{\Lcal}{\mathcal{L}}
\newcommand{\Mcal}{\mathcal{M}}
\newcommand{\Ncal}{\mathcal{N}}
\newcommand{\Ocal}{\mathcal{O}}
\newcommand{\Pcal}{\mathcal{P}}
\newcommand{\Qcal}{\mathcal{Q}}
\newcommand{\Rcal}{\mathcal{R}}
\newcommand{\Scal}{\mathcal{S}}
\newcommand{\Tcal}{\mathcal{T}}
\newcommand{\Ucal}{\mathcal{U}}
\newcommand{\Vcal}{\mathcal{V}}
\newcommand{\Wcal}{\mathcal{W}}
\newcommand{\Xcal}{\mathcal{X}}
\newcommand{\Ycal}{\mathcal{Y}}
\newcommand{\Zcal}{\mathcal{Z}}
\newcommand{\Ascr}{\mathscr{A}}
\newcommand{\Bscr}{\mathscr{B}}
\newcommand{\Cscr}{\mathscr{C}}
\newcommand{\Dscr}{\mathscr{D}}
\newcommand{\Escr}{\mathscr{E}}
\newcommand{\Fscr}{\mathscr{F}}
\newcommand{\Gscr}{\mathscr{G}}
\newcommand{\Hscr}{\mathscr{H}}
\newcommand{\Iscr}{\mathscr{I}}
\newcommand{\Jscr}{\mathscr{J}}
\newcommand{\Kscr}{\mathscr{K}}
\newcommand{\Lscr}{\mathscr{L}}
\newcommand{\Mscr}{\mathscr{M}}
\newcommand{\Nscr}{\mathscr{N}}
\newcommand{\Oscr}{\mathscr{O}}
\newcommand{\Pscr}{\mathscr{P}}
\newcommand{\Qscr}{\mathscr{Q}}
\newcommand{\Rscr}{\mathscr{R}}
\newcommand{\Sscr}{\mathscr{S}}
\newcommand{\Tscr}{\mathscr{T}}
\newcommand{\Uscr}{\mathscr{U}}
\newcommand{\Vscr}{\mathscr{V}}
\newcommand{\Wscr}{\mathscr{W}}
\newcommand{\Xscr}{\mathscr{X}}
\newcommand{\Yscr}{\mathscr{Y}}
\newcommand{\Zscr}{\mathscr{Z}}
\newcommand{\Afrak}{\mathfrak{A}}
\newcommand{\Bfrak}{\mathfrak{B}}
\newcommand{\Cfrak}{\mathfrak{C}}
\newcommand{\Dfrak}{\mathfrak{D}}
\newcommand{\Efrak}{\mathfrak{E}}
\newcommand{\Ffrak}{\mathfrak{F}}
\newcommand{\Gfrak}{\mathfrak{G}}
\newcommand{\Hfrak}{\mathfrak{H}}
\newcommand{\Ifrak}{\mathfrak{I}}
\newcommand{\Jfrak}{\mathfrak{J}}
\newcommand{\Kfrak}{\mathfrak{K}}
\newcommand{\Lfrak}{\mathfrak{L}}
\newcommand{\Mfrak}{\mathfrak{M}}
\newcommand{\Nfrak}{\mathfrak{N}}
\newcommand{\Ofrak}{\mathfrak{O}}
\newcommand{\Pfrak}{\mathfrak{P}}
\newcommand{\Qfrak}{\mathfrak{Q}}
\newcommand{\Rfrak}{\mathfrak{R}}
\newcommand{\Sfrak}{\mathfrak{S}}
\newcommand{\Tfrak}{\mathfrak{T}}
\newcommand{\Ufrak}{\mathfrak{U}}
\newcommand{\Vfrak}{\mathfrak{V}}
\newcommand{\Wfrak}{\mathfrak{W}}
\newcommand{\Xfrak}{\mathfrak{X}}
\newcommand{\Yfrak}{\mathfrak{Y}}
\newcommand{\Zfrak}{\mathfrak{Z}}
\newcommand{\afrak}{\mathfrak{a}}
\newcommand{\bfrak}{\mathfrak{b}}
\newcommand{\cfrak}{\mathfrak{c}}
\newcommand{\dfrak}{\mathfrak{d}}
\newcommand{\efrak}{\mathfrak{e}}
\newcommand{\ffrak}{\mathfrak{f}}
\newcommand{\gfrak}{\mathfrak{g}}
\newcommand{\hfrak}{\mathfrak{h}}
\newcommand{\ifrak}{\mathfrak{i}}
\newcommand{\jfrak}{\mathfrak{j}}
\newcommand{\kfrak}{\mathfrak{k}}
\newcommand{\lfrak}{\mathfrak{l}}
\newcommand{\mfrak}{\mathfrak{m}}
\newcommand{\nfrak}{\mathfrak{n}}
\newcommand{\ofrak}{\mathfrak{o}}
\newcommand{\pfrak}{\mathfrak{p}}
\newcommand{\qfrak}{\mathfrak{q}}
\newcommand{\rfrak}{\mathfrak{r}}
\newcommand{\sfrak}{\mathfrak{s}}
\newcommand{\tfrak}{\mathfrak{t}}
\newcommand{\ufrak}{\mathfrak{u}}
\newcommand{\vfrak}{\mathfrak{v}}
\newcommand{\wfrak}{\mathfrak{w}}
\newcommand{\xfrak}{\mathfrak{x}}
\newcommand{\yfrak}{\mathfrak{y}}
\newcommand{\zfrak}{\mathfrak{z}}
\newcommand{\Abar}{\overline{A}}
\newcommand{\Bbar}{\overline{B}}
\newcommand{\Cbar}{\overline{C}}
\newcommand{\Dbar}{\overline{D}}
\newcommand{\Ebar}{\overline{E}}
\newcommand{\Fbar}{\overline{F}}
\newcommand{\Gbar}{\overline{G}}
\newcommand{\Hbar}{\overline{H}}
\newcommand{\Ibar}{\overline{I}}
\newcommand{\Jbar}{\overline{J}}
\newcommand{\Kbar}{\overline{K}}
\newcommand{\Lbar}{\overline{L}}
\newcommand{\Mbar}{\overline{M}}
\newcommand{\Nbar}{\overline{N}}
\newcommand{\Obar}{\overline{O}}
\newcommand{\Pbar}{\overline{P}}
\newcommand{\Qbar}{\overline{Q}}
\newcommand{\Rbar}{\overline{R}}
\newcommand{\Sbar}{\overline{S}}
\newcommand{\Tbar}{\overline{T}}
\newcommand{\Ubar}{\overline{U}}
\newcommand{\Vbar}{\overline{V}}
\newcommand{\Wbar}{\overline{W}}
\newcommand{\Xbar}{\overline{X}}
\newcommand{\Ybar}{\overline{Y}}
\newcommand{\Zbar}{\overline{Z}}
\newcommand{\abar}{\overline{a}}
\newcommand{\bbar}{\overline{b}}
\newcommand{\cbar}{\overline{c}}
\newcommand{\dbar}{\overline{d}}
\newcommand{\ebar}{\overline{e}}
\newcommand{\fbar}{\overline{f}}
\newcommand{\gbar}{\overline{g}}
\renewcommand{\hbar}{\overline{h}}
\newcommand{\ibar}{\overline{i}}
\newcommand{\jbar}{\overline{j}}
\newcommand{\kbar}{\overline{k}}
\newcommand{\lbar}{\overline{l}}
\newcommand{\mbar}{\overline{m}}
\newcommand{\nbar}{\overline{n}}
\newcommand{\obar}{\overline{o}}
\newcommand{\pbar}{\overline{p}}
\newcommand{\qbar}{\overline{q}}
\newcommand{\rbar}{\overline{r}}
\newcommand{\sbar}{\overline{s}}
\newcommand{\tbar}{\overline{t}}
\newcommand{\ubar}{\overline{u}}
\newcommand{\vbar}{\overline{v}}
\newcommand{\wbar}{\overline{w}}
\newcommand{\xbar}{\overline{x}}
\newcommand{\ybar}{\overline{y}}
\newcommand{\zbar}{\overline{z}}
\newcommand\bigzero{\makebox(0,0){\text{\huge0}}}
\newcommand{\limp}{\lim\limits_{\leftarrow}}
\newcommand{\limi}{\lim\limits_{\rightarrow}}


\DeclareMathOperator{\End}{\mathrm{End}}
\DeclareMathOperator{\Hom}{\mathrm{Hom}}
\DeclareMathOperator{\Vect}{\mathrm{Vect}}
\DeclareMathOperator{\Spec}{\mathrm{Spec}}
\DeclareMathOperator{\multideg}{\mathrm{multideg}}
\DeclareMathOperator{\LM}{\mathrm{LM}}
\DeclareMathOperator{\LT}{\mathrm{LT}}
\DeclareMathOperator{\LC}{\mathrm{LC}}
\DeclareMathOperator{\PPCM}{\mathrm{PPCM}}
\DeclareMathOperator{\PGCD}{\mathrm{PGCD}}
\DeclareMathOperator{\Syl}{\mathrm{Syl}}
\DeclareMathOperator{\Res}{\mathrm{Res}}
\DeclareMathOperator{\Com}{\mathrm{Com}}
\DeclareMathOperator{\GL}{\mathrm{GL}}
\DeclareMathOperator{\SL}{\mathrm{SL}}
\DeclareMathOperator{\SU}{\mathrm{SU}}
\DeclareMathOperator{\PGL}{\mathrm{PGL}}
\DeclareMathOperator{\PSL}{\mathrm{PSL}}
\DeclareMathOperator{\PSU}{\mathrm{PSU}}
\DeclareMathOperator{\SO}{\mathrm{SO}}
\DeclareMathOperator{\Sp}{\mathrm{Sp}}
\DeclareMathOperator{\Spin}{\mathrm{Spin}}
\DeclareMathOperator{\Ker}{\mathrm{Ker}}
%\DeclareMathOperator{\Im}{\mathrm{Im}}

\DeclareMathOperator{\Ens}{\mathbf{Ens}}
\DeclareMathOperator{\Top}{\mathbf{Top}}
\DeclareMathOperator{\Ann}{\mathbf{Ann}}
\DeclareMathOperator{\Gr}{\mathbf{Gr}}
\DeclareMathOperator{\Ab}{\mathbf{Ab}}
%\DeclareMathOperator{\Vect}{\mathbf{Vect}}
\DeclareMathOperator{\Mod}{\mathbf{Mod}}

\newcommand{\bishop}{bishop_conditions_1964}
\newcommand{\cac}{barlet_cycles_2014}
\newcommand{\btg}{bourbaki_topologie_2007}   
\newcommand{\pag}{griffiths_principles_2011}
\newcommand{\barlet}{barlet_espace_1975}
\newcommand{\voisin}{voisin_theorie_2002}
\newcommand{\scv}{grauert_several_2013}
\newcommand{\bredon}{bredon_topology_1993}
\newcommand{\BM}{bochner_groups_1947}
\newcommand{\Lee}{lee_structure_2001}
\newcommand{\laf}{lafontaine_introduction_2012}
\newcommand{\ccs}{barth_compact_2004}
\newcommand{\cas}{grauert_coherent_1984}
\newcommand{\akh}{akhiezer_lie_2012}
\newcommand{\catan}{catanese_moduli_1988}
\newcommand{\Hart}{hartshorne_algebraic_1977}
\newcommand{\Hatcher}{hatcher_algebraic_2002}

%\headheight=0mm
%\topmargin=-20mm
\oddsidemargin=31pt
\evensidemargin=31pt
%\textwidth=18cm
%\textheight=25cm
%\setlength{\marginparwidth}{0pt}

\parindent=0mm
\newif\ifproof
\newcommand{\demo}[1]{\ifproof #1 \else \fi}
 %Instruction d'utilisation : 
%les preuves du texte sont, en principe, entre des balises \demo, en sus des \begin{proof} pour l'instant.
%Laisser le texte tel quel, fait qu'elles ne sont pas affich�es.
%Mettre \prooftrue fait que toutes les preuves jusqu'� un \prooffalse ou la fin du document. 


 \begin{document}
\newtheorem{Thm}{Th�or�me}[chapter]
\newtheorem{Prop}[Thm]{Proposition}
\newtheorem{Propte}[Thm]{Propri�t�}
\newtheorem{Lemme}[Thm]{Lemme}
\newtheorem{Cor}[Thm]{Corollaire}


\theoremstyle{definition}

\newtheorem{Ex}[Thm]{Exemple}
\newtheorem{Def}[Thm]{D�finition}
\newtheorem{Defpropte}[Thm]{D�finition et propri�t�}
\newtheorem{Defprop}[Thm]{D�finition et proposition}
\newtheorem{Defthm}[Thm]{Th�or�me et d�finition}
\newtheorem{Not}[Thm]{Notation}
\newtheorem{Conv}[Thm]{Convention}
\newtheorem{Const}[Thm]{Construction}

\theoremstyle{remark}
\newtheorem{Rq}[Thm]{Remarque}
\newtheorem{Slog}[Thm]{Slogan}
\newtheorem{Exo}[Thm]{Exercice}

\fi


Soit $f : X \to Y$ une fonction holomorphe �quivariante pour une action de $H$ sur $X$ et $Y$. Supposons que $H$ agit de fa�on compactifiable sur $X$ et que $Y$ est de K�hler. \\
L'�quivariance de $f$ fait que l'automorphisme induit par $h \in H$ conserve $f(X)$ i.e. on a une morphisme de groupes holomorphe $H \to Aut_0(Y)_{\{f(X)\}}$ que l'on peut ensuite composer avec la projection canonique $Aut_0(Y)_{\{f(X)\}} \to Aut_0(Y)_{\{f(X)\}} /Aut_0(Y)_{f(X)}=:G$ pour obtenir un morphisme $\lambda : H \to G$

\begin{Prop} \label{image_noyau_Zar}
Le noyau de $\lambda$  est $\Zcal$-ferm� et l'image de $\lambda$ est un ferm� de $G$ pour la ($\Zcal$-)topologie quotient.
\end{Prop}
\begin{proof}
  Montrons que $\Ker(\lambda)$ est $\Zcal$-ferm� :
  Pour cela, il suffit de montrer l'�galit� :
   \[
    \Ker(\lambda)=\bigcap_{y \in f(X)}H_{\{f^{-1}(y)\}}
  \]
  et d'utiliser la proposition \ref{}. \\
  On a, par d�finition,
  \[
    \Ker(\lambda)=\{ h \in H \mid \forall y\in f(X), h \cdot y=y \}
  \]
  Soit $h \in \Ker(\lambda)$. Soient $y \in f(X)$ et $x \in f^{-1}(y)$. Alors,
  \[
    f(h \cdot x)=h \cdot f(x)=h \cdot y=y
  \]
  i.e $h \cdot x \in f^{-1}(y)$. \\
  Le m�me raisonnement pour $h^{-1}$ nous montre que $h \in H_{\{f^{-1}(y)\}}$.    \\
  Cette �galit� �tant vraie pour tout $y \in Y$, on obtient l'inclusion :  $\Ker(\lambda) \subset \bigcap_{y \in f(X)}H_{\{f^{-1}(y)\}}$. \\
  R�ciproquement, si $h \in \bigcap_{y \in f(X)}H_{\{f^{-1}(y)\}}$, alors pour tout $y \in f(X)$ et $x \in X$ tel que $y=f(x)$, \\
  \[
    h \cdot y=h\cdot f(x)=f(h \cdot x)=y
  \]
  car $x \in f^{-1}(y)$. Cela montre que $ h \in Ker(\lambda)$, ce qui montre l'�galit� voulue. \\
  Montrons maintenant que $Im(\lambda)$ est un ferm� de Zariski. \\
  On peut commencer par remarquer que $G$ agit de fa�on compactifiable sur $f(X)$. En effet, par \ref{}, $Aut_0(Y)$ agit de fa�on compactifiable sur $Y$. Comme $Aut_0(Y)_{\{f(X)\}}$ est un $\Zcal$-ferm� (par \ref{} et le fait que $f(X)$ est un ferm� de Zariski car $f$ est propre) alors $Aut_0(Y)_{\{f(X)\}}$ agit de fa�on compactifiable sur $Y$ et en particulier sur $f(X)$. Comme $G$ est le quotient de $Aut_0(Y)_{\{f(X)\}}$ par le sous-groupe d'ineffectivit�, l'action passe au quotient et on obtient une action de $G$ sur $f(X)$ compactifiable. \\
  D'apr�s \ref{}, on peut choisir $y_1,\ldots,y_n \in f(X)$ de tel sorte que l'application $\varphi : G \to Y^n$ d�finie , pour tout $h \in H$, par :
  \[
    \varphi(g)=(g\cdot y_1,\ldots,g \cdot y_n)
  \]
  permette d'identifier $G$ avec un ouvert de Zariski de $\overline{\varphi(G)}$
  Soit $x_i$ tel que $y_i=f(x_i)$. Par la m�me proposition, on obtient une application 
		\[\psi : h\in H \mapsto (h\cdot x_1,\ldots,h\cdot x_n) \in X^n
	\]
	 qui identifie $H$ avec un ouvert de Zariski de $\overline{\psi(H)}$. \\
  Comme $f$ est propre alors $f \times \ldots \times f(\overline{\psi(H)})$ est un ferm� de Zariski et par \ref{}, $f \times \ldots \times f(\psi(H))$ est un ensemble constructible de $f \times \ldots \times f( \overline{\psi(H)})$. On peut ensuite remarquer que le diagramme: 
 
	 \[\xymatrix{
			 H \ar[r]^\varphi \ar[d]_\lambda  & X^n \ar[d]^{f \times \cdot \times f} \\
			G \ar[r]^\psi & Y^n
		}
\]
  est commutatif. En effet, pour tout $h \in H$,
  \[
    (f \times \cdot \times f) \circ \psi(h)= (f \times \cdot \times f)(h\cdot x_1,\ldots,h \cdot x_n)=(f(h \cdot x_1),\cdots, f(h \cdots x_n))=(h\cdot y_1,\ldots,h \cdot y_n)
  \]
  et
  \[
    \varphi(\lambda(h))=\varphi(h : x \mapsto h \cdot x)=(h\cdot y_1,\ldots,h \cdot y_n)
  \]
  Par cons�quent, $f \times \ldots \times f(\psi(H)})$ est inclus dans $ \varphi(G) $. Ce qui veut dire que $f \times \ldots \times f(\psi(H)})$ est une partie constructible de $\varphi(G)$. L'identification donn�e par $\varphi$ munit $\varphi(G)$ d'une structure de groupe topologique pour laquelle $f \times \ldots \times f(\psi(H)})$ est un sous-groupe. Par \ref{}, $f \times \ldots \times f(\psi(H)})$ est un sous-groupe ferm� (pour la topologie de Zariski) et comme $\lambda(H)=\varphi^{-1}(f \times \ldots \times f(\psi(H)})$ alors $\lambda(H)$ est un ferm� de Zariski.

\end{proof}


\begin{Ex}
  Soient $f$, $X$ et $Y$ comme dans \ref{image_noyau_Zar}, $H$ un sous-groupe Zariski-ferm� de $Aut_0(X)$ et soit $\lambda : H \to Aut_0(Y)$ d�finie par $h \mapsto (y \mapsto g(y))$. Alors $\lambda(H)$ n'est pas n�cessairement $\Zcal$-ferm� dans $Aut_0(Y)$ : \\
  Soient $X=\P^1$, $Y=X \times \P^1$ et $H=\C$. $H$ agit sur $X$ par :
  \[t \cdot [x:y] = \begin{pmatrix} 1 & t \\ 0 & 1 \end{pmatrix}[x:y]=[x+ty : y]\]
  et sur la deuxi�me composante de $Y$ par :
    \[t \cdot [x:y] = \begin{pmatrix} 1 & 0 \\ 0 & e^t \end{pmatrix}[x:y]=[x :e^t y]\]
    On peut commencer par remarquer que $H$ agit de fa�on compactifiable sur $X$ avec l'injection $\C \to \P^1$ et l'action m�romorphe donn�e pour $x \in \P^1$ par :
    \[
      [1:0] \cdot x=[1:0]
    \]
    car, si $x_2 \neq 0$, $\lim \limits_{t \to \infty}x_1+tx_2=\infty$ et si $x_2=0$ alors pour tout $t \in \C$, $t \cdot [x_1,0]=[1:0]$.
    Et que $Y$ est de K�hler comme produit de vari�t�s de K�hler et que $Aut_0(Y)=Aut_0(\P^1) \times Aut_0(\P^1)=PGL_2(\C)^2$. \\
    On obtient donc que :
    \[\lambda(H)=\left\{\left(\begin{pmatrix} 1 & t \\ 0 & 1 \end{pmatrix},\begin{pmatrix}1 & 0 \\ 0 &e^t \end{pmatrix}\right) \mid t \in \R \right\}
    \]
    qui n'est pas $\Zcal$-ferm� car la $\Zcal$-topologie sur $PGL_n(\C)$ est la topologie de Zariski alg�brique (voir \ref{}) et la pr�sence de l'exponentielle emp�che que cela soit alg�brique.
\end{Ex}


\ifwhole
 \end{document}
\fi