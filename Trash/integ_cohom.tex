\subsection{ Int�grations et classes de cohomologie}
\begin{Thm}[Lemme de Poincar�]
Soit $M$ une vari�t� diff�rentielle orient�e et soit $k$ un entier, alors on a un isomorphisme
	\[ H^k_{dR}(M) \simeq H^{n-k}_{dR}(M)^*
\]
donn� par $[\varphi] \mapsto ([\psi] \mapsto \int_M \varphi \wedge \psi)$.

\end{Thm}
\begin{proof}
\cite{\pag}
\end{proof}
Soit $M$ une vari�t� complexe compacte, $V \subset M$ un sous-ensemble analytique de dimension $k$. On peut d�finir une application lin�aire :
	\[ H^{2k}_{dR}(M) \to \R, [\varphi] \mapsto \int_V \varphi
	\]
Cette application est bien d�finie car les formes exactes sont envoy�es sur 0 par le th�or�me de Stokes : 	 
\begin{Thm}[Stokes pour les espaces analytiques]
Soit $M$ une vari�t� complexe, $V$ un sous-ensemble analytique de dimension $k$, et $\varphi$ une forme diff�rentielle de degr� $2k-1$ � support compact de $M$,
	\[\int_V d\varphi=0
\]

\end{Thm}
\begin{proof}
cf \cite{\pag} p33
\end{proof}

Par le lemme de Poincar� \ref{lem_poinc}, cette application d�termine un �l�ment de $H^{2n-2k}_{dR}(M)$ que l'on appelera classe (fondamentale) de $V$ et que l'on notera $[V]$. On d�finit la classe $[Z]$ d'un cycle analytique $Z=\sum_i n_i Z_i$ d'une vari�t� $M$ comme �tant $\sum n_i [Z_i]$

Dans le cas o� $(M,\omega)$ est une vari�t� de K�hler, on peut d�finir le volume d'une classe de cohomologie qui g�n�ralise celui des sous-espaces analytiques :
\begin{Def}
Soit $\alpha \in H^k_{dR}(M)$. On appelle volume de $\alpha$ le r�el $Vol(\alpha)=\frac{1}{n !}\int_M \alpha \wedge \omega^{n-k}$.
\end{Def}
Si $V$ est un sous-espace analytique de $M$ alors par d�finition de la classe de $V$, 
	\[Vol([V])=\frac{1}{n !}\int_V \omega^{n-k}=Vol(V)
\]
De plus, si $Z=\sum_i n_i Z_i$ est un cycle  de $M$, $Vol([Z])=\sum n_i Vol([Z_i])=\sum n_i Vol(Z_i)$ est appel� volume du cycle $Z$.