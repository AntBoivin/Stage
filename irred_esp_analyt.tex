\begin{Def}
Soit $X$ un espace analytique complexe r�duit.
On dira que $X$ est irr�ductible en un point $x$ de $X$ si $x$ admet un syst�me fondamental de voisinages ouverts irr�ductibles dans $X$ et que $X$ est localement irr�ductible s'il est irr�ductible en tous ses points.
\end{Def}
En particulier, il existe un recouvrement de $X$ par des ouverts irr�ductibles.
\begin{Prop} \label{connex_irred}
Un espace topologique localement irr�ductible est irr�ductible si, et seulement si, il est connexe.

\end{Prop}
\begin{proof}
  $\Leftarrow$ :
  Soit $X$ un espace topologique localement irr�ductible connexe. \\
  Soient $\{U_i\}$ un recouvrement de $X$ par des ouverts irr�ductibles et $V,W$ deux ouverts de $X$ disjoints. \\
  Montrons, tout d'abord, qu'un ouvert $U_i$ rencontre soit $V$ soit $W$ soit aucun des deux : \\
  Supposons, par l'absurde, qu'il existe $U_i$ tel que $U_i \cap V \neq \emptyset \neq U_i \cap W$. Comme $U$ est irr�ductible et que $U_i \cap V$ et $U_i \cap W$ sont des ouverts de $U_i$ alors $U_i\cap V=\emptyset$ ou $U_i \cap W=\emptyset$, ce qui montre ce que l'on voulait. \\
  On peut donc trier les ouverts du recouvrement $\{U_i\}$ en trois familles : ceux qui rencontrent $V$, ceux qui rencontrent $W$ et ceux qui rencontrent ni l'un ni l'autre. Notons par $A$, $B$ et $C$ l'union de, respectivement, des ouverts de la premi�re famille, ceux de la deuxi�me et ceux de la troisi�me. On a ainsi �crit $X$ comme une union disjointe de trois ouverts :
  \[ X=A \coprod B \coprod C\]
  Comme $X$ est connexe alors $A \coprod B$ ou $C$ est vide :
  \begin{itemize}
  \item Si $A \coprod B$ est vide alors $V$ et $W$ le sont aussi.
  \item Si $C$ est vide alors $X=A \coprod B$. Comme $X$ est connexe alors $A$ ou $B$ est vide. Et donc $V$ ou $W$ est vide.
  \end{itemize}
  Ce qui montre que $X$ est irr�ductible. \\
  $\Rightarrow$ : Soit $X$ un espace topologique irr�ductible. \\
  Soit $U,V$ deux ouverts disjoints tel que $X=U \cup V$. \\
  Comme $U$ et $V$ sont disjoints et que $X$ est irr�ductible alors $U$ ou $V$ est vide. Ce qui montre que $X$ est connexe.
\end{proof}
\begin{Prop} \label{irred_fibre}
Soient $X$ un espace analytique complexe r�duit et $x \in X$. Alors $X$ est irr�ductible en $x$ si, et seulement si, l'anneau $\Ocal_{X,x}$ est int�gre.
\end{Prop}
\begin{proof}
Voir \cite{\cac} p171
\end{proof}
\begin{Cor} \label{normal_irred}
Un espace analytique r�duit, connexe et normal est irr�ductible.
\end{Cor}
\begin{proof}
Soit $X$ un espace analytique r�duit, connexe et normal.
Comme $X$ est normal alors en tout point $x \in X$, $\Ocal_{X,x}$ est int�gralement clos et en particulier est int�gre. Autrement dit, par la proposition \ref{irred_fibre}, $X$ est irr�ductible en tout ses points ou encore $X$ est localement irr�ductible. Comme, de plus, $X$ est connexe alors, par \ref{connex_irred}, $X$ est irr�ductible.
\end{proof}