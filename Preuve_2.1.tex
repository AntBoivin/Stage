\begin{proof} 
 On supposera $X$ lisse. \\
Soit $Z \subset X \times X \times \Cscr_\Gamma$ le cycle universel.  La projection naturelle $\pi_3 : Z \to \Cscr_\Gamma$ est propre (cf Barlet).\\
Quitte � remplacer $\Cscr_\Gamma$ par un ouvert de Zariski, on peut supposer $\Cscr_\Gamma$ et $Z$ lisse. En effet, les lieux singuliers de $\Cscr_\Gamma$ et $Z$ sont des espaces analytiques. L'image par $\pi_3$ de ce dernier est aussi un espace analytique et donc son compl�mentaire est un ouvert de Zariski. L'image r�ciproque de cet ouvert est donc lisse.
Par la proposition 1.21 p 108 SVC, il existe un ouvert de Zariski $U$ tel que $\pi_3_{|U}$ soit une submersion. Par les m�mes consid�rations que pour la lissit� de $Z$, $\pi_3$ est lisse sur un ouvert de Zariski de $\Cscr_\Gamma$. Ainsi, toutes les fibres sur cet ouvert sont lisses.
De la m�me fa�on, pour les projections $\pi_i : Z \to X$, on obtient que,sur un ouvert de Zariski $S_i$ de $Z$, ce sont des submersions. En tout point de cet ouvert, on a un isomorphisme entre le quotient  $T_x Z$ par $T_{x} \pi_i^{-1}(\pi_i(x))}$ et $T_{f(x)} X$ et donc un isomorphisme entre $T_z Z_C$ et $T_{\pi_i(z)} X$ pour tout $z=(x,y,C) \in Z$  tel que $(x,y) \in A$. \\
Pour finir, montrons que les points de $\Cscr_\Gamma \setminus \pi_3(S_1 \cup S_2)$ correspondent aux automorphismes de $X$. \\
Soit $C$ un cycle de $\Cscr_\Gamma$. \\
Les applications $\pi_i : Z_C \to X$ est un rev�tement propre par \ref{}. \\
Comme $\Cscr_\Gamma$ est connexe et localement connexe par arcs alors il est connexe par arcs. Il existe donc un chemin $\gamma : [0,1] \to \Cscr_\Gamma$ entre $C$ et $\Gamma$. Par continuit� de la famille $(\gamma(t))_{t \in [0,1]}$ le degr� du rev�tement reste constant ( au voisinage de chacun des points de $X$) et donc vaut toujours 1. $\pi_i$ est donc un hom�omorphisme et donc par l'isomorphisme pr�c�dent et le th�or�me d'inversion locale, c'est un biholomorphisme.

\end{proof}