

\newif\ifwhole

%\wholetrue
% Ajouter \wholetrue si on compile seulement ce fichier

\ifwhole
 \documentclass[a4page,10pt]{article}
     \usepackage[Latin1]{inputenc}
\usepackage[francais]{babel}
\usepackage{amsmath,amssymb,amsthm}
\usepackage{textcomp}
\usepackage{mathrsfs}
\usepackage{algcompatible,algorithm  }
\usepackage[all]{xy}
\usepackage{hyperref}
\usepackage{fancyhdr}
\usepackage{supertabular}
\usepackage{makeidx}
\usepackage{setspace}
\usepackage{makeidx}
\usepackage[Bjornstrup]{fncychap}
\usepackage[nottoc,notlot,notlof]{tocbibind}
\makeindex
\pagestyle{fancy}
\fancyhead[L]{\leftmark}
\fancyhead[R]{}
\onehalfspacing

\toks0=\expandafter{\xy}
\edef\xy{\noexpand\shorthandoff{!?;:}\the\toks0 }
\makeatletter

\renewcommand*{\ALG@name}{Algorithme}

\makeatother

\renewcommand{\algorithmicrequire}{\textbf{\textsc {Entr\'ees  :  } } }
\renewcommand{\algorithmicensure}{\textbf{\textsc { Sortie  :  } } }
\renewcommand{\algorithmicwhile}{\textbf{Tant que}}
\renewcommand{\algorithmicdo}{\textbf{faire }}
\renewcommand{\algorithmicif}{\textbf{Si}}
\renewcommand{\algorithmicelse}{\textbf{Sinon}}
\renewcommand{\algorithmicthen}{\textbf{alors }}
\renewcommand{\algorithmicend}{\textbf{fin}}
\renewcommand{\algorithmicfor}{\textbf{Pour}}
\renewcommand{\algorithmicuntil}{\textbf{Jusqu'\`a}}
\renewcommand{\algorithmicrepeat}{\textbf{Répéter}}

\newcommand{\A}{\mathbb{A}}
\newcommand{\B}{\mathbb{B}}
\newcommand{\C}{\mathbb{C}}
\newcommand{\D}{\mathbb{D}}
\newcommand{\E}{\mathbb{E}}
\newcommand{\F}{\mathbb{F}}
\newcommand{\G}{\mathbb{G}}
\renewcommand{\H}{\mathbb{H}}
\newcommand{\I}{\mathbb{I}}
\newcommand{\J}{\mathbb{J}}
\newcommand{\K}{\mathbb{K}}
\renewcommand{\L}{\mathbb{L}}
\newcommand{\M}{\mathbb{M}}
\newcommand{\N}{\mathbb{N}}
\renewcommand{\O}{\mathbb{O}}
\renewcommand{\P}{\mathbb{P}}
\newcommand{\Q}{\mathbb{Q}}
\newcommand{\R}{\mathbb{R}}
\renewcommand{\S}{\mathbb{S}}
\newcommand{\T}{\mathbb{T}}
\newcommand{\U}{\mathbb{U}}
\newcommand{\V}{\mathbb{V}}
\newcommand{\W}{\mathbb{W}}
\newcommand{\X}{\mathbb{X}}
\newcommand{\Y}{\mathbb{Y}}
\newcommand{\Z}{\mathbb{Z}}
\newcommand{\Acal}{\mathcal{A}}
\newcommand{\Bcal}{\mathcal{B}}
\newcommand{\Ccal}{\mathcal{C}}
\newcommand{\Dcal}{\mathcal{D}}
\newcommand{\Ecal}{\mathcal{E}}
\newcommand{\Fcal}{\mathcal{F}}
\newcommand{\Gcal}{\mathcal{G}}
\newcommand{\Hcal}{\mathcal{H}}
\newcommand{\Ical}{\mathcal{I}}
\newcommand{\Jcal}{\mathcal{J}}
\newcommand{\Kcal}{\mathcal{K}}
\newcommand{\Lcal}{\mathcal{L}}
\newcommand{\Mcal}{\mathcal{M}}
\newcommand{\Ncal}{\mathcal{N}}
\newcommand{\Ocal}{\mathcal{O}}
\newcommand{\Pcal}{\mathcal{P}}
\newcommand{\Qcal}{\mathcal{Q}}
\newcommand{\Rcal}{\mathcal{R}}
\newcommand{\Scal}{\mathcal{S}}
\newcommand{\Tcal}{\mathcal{T}}
\newcommand{\Ucal}{\mathcal{U}}
\newcommand{\Vcal}{\mathcal{V}}
\newcommand{\Wcal}{\mathcal{W}}
\newcommand{\Xcal}{\mathcal{X}}
\newcommand{\Ycal}{\mathcal{Y}}
\newcommand{\Zcal}{\mathcal{Z}}
\newcommand{\Ascr}{\mathscr{A}}
\newcommand{\Bscr}{\mathscr{B}}
\newcommand{\Cscr}{\mathscr{C}}
\newcommand{\Dscr}{\mathscr{D}}
\newcommand{\Escr}{\mathscr{E}}
\newcommand{\Fscr}{\mathscr{F}}
\newcommand{\Gscr}{\mathscr{G}}
\newcommand{\Hscr}{\mathscr{H}}
\newcommand{\Iscr}{\mathscr{I}}
\newcommand{\Jscr}{\mathscr{J}}
\newcommand{\Kscr}{\mathscr{K}}
\newcommand{\Lscr}{\mathscr{L}}
\newcommand{\Mscr}{\mathscr{M}}
\newcommand{\Nscr}{\mathscr{N}}
\newcommand{\Oscr}{\mathscr{O}}
\newcommand{\Pscr}{\mathscr{P}}
\newcommand{\Qscr}{\mathscr{Q}}
\newcommand{\Rscr}{\mathscr{R}}
\newcommand{\Sscr}{\mathscr{S}}
\newcommand{\Tscr}{\mathscr{T}}
\newcommand{\Uscr}{\mathscr{U}}
\newcommand{\Vscr}{\mathscr{V}}
\newcommand{\Wscr}{\mathscr{W}}
\newcommand{\Xscr}{\mathscr{X}}
\newcommand{\Yscr}{\mathscr{Y}}
\newcommand{\Zscr}{\mathscr{Z}}
\newcommand{\Afrak}{\mathfrak{A}}
\newcommand{\Bfrak}{\mathfrak{B}}
\newcommand{\Cfrak}{\mathfrak{C}}
\newcommand{\Dfrak}{\mathfrak{D}}
\newcommand{\Efrak}{\mathfrak{E}}
\newcommand{\Ffrak}{\mathfrak{F}}
\newcommand{\Gfrak}{\mathfrak{G}}
\newcommand{\Hfrak}{\mathfrak{H}}
\newcommand{\Ifrak}{\mathfrak{I}}
\newcommand{\Jfrak}{\mathfrak{J}}
\newcommand{\Kfrak}{\mathfrak{K}}
\newcommand{\Lfrak}{\mathfrak{L}}
\newcommand{\Mfrak}{\mathfrak{M}}
\newcommand{\Nfrak}{\mathfrak{N}}
\newcommand{\Ofrak}{\mathfrak{O}}
\newcommand{\Pfrak}{\mathfrak{P}}
\newcommand{\Qfrak}{\mathfrak{Q}}
\newcommand{\Rfrak}{\mathfrak{R}}
\newcommand{\Sfrak}{\mathfrak{S}}
\newcommand{\Tfrak}{\mathfrak{T}}
\newcommand{\Ufrak}{\mathfrak{U}}
\newcommand{\Vfrak}{\mathfrak{V}}
\newcommand{\Wfrak}{\mathfrak{W}}
\newcommand{\Xfrak}{\mathfrak{X}}
\newcommand{\Yfrak}{\mathfrak{Y}}
\newcommand{\Zfrak}{\mathfrak{Z}}
\newcommand{\afrak}{\mathfrak{a}}
\newcommand{\bfrak}{\mathfrak{b}}
\newcommand{\cfrak}{\mathfrak{c}}
\newcommand{\dfrak}{\mathfrak{d}}
\newcommand{\efrak}{\mathfrak{e}}
\newcommand{\ffrak}{\mathfrak{f}}
\newcommand{\gfrak}{\mathfrak{g}}
\newcommand{\hfrak}{\mathfrak{h}}
\newcommand{\ifrak}{\mathfrak{i}}
\newcommand{\jfrak}{\mathfrak{j}}
\newcommand{\kfrak}{\mathfrak{k}}
\newcommand{\lfrak}{\mathfrak{l}}
\newcommand{\mfrak}{\mathfrak{m}}
\newcommand{\nfrak}{\mathfrak{n}}
\newcommand{\ofrak}{\mathfrak{o}}
\newcommand{\pfrak}{\mathfrak{p}}
\newcommand{\qfrak}{\mathfrak{q}}
\newcommand{\rfrak}{\mathfrak{r}}
\newcommand{\sfrak}{\mathfrak{s}}
\newcommand{\tfrak}{\mathfrak{t}}
\newcommand{\ufrak}{\mathfrak{u}}
\newcommand{\vfrak}{\mathfrak{v}}
\newcommand{\wfrak}{\mathfrak{w}}
\newcommand{\xfrak}{\mathfrak{x}}
\newcommand{\yfrak}{\mathfrak{y}}
\newcommand{\zfrak}{\mathfrak{z}}
\newcommand{\Abar}{\overline{A}}
\newcommand{\Bbar}{\overline{B}}
\newcommand{\Cbar}{\overline{C}}
\newcommand{\Dbar}{\overline{D}}
\newcommand{\Ebar}{\overline{E}}
\newcommand{\Fbar}{\overline{F}}
\newcommand{\Gbar}{\overline{G}}
\newcommand{\Hbar}{\overline{H}}
\newcommand{\Ibar}{\overline{I}}
\newcommand{\Jbar}{\overline{J}}
\newcommand{\Kbar}{\overline{K}}
\newcommand{\Lbar}{\overline{L}}
\newcommand{\Mbar}{\overline{M}}
\newcommand{\Nbar}{\overline{N}}
\newcommand{\Obar}{\overline{O}}
\newcommand{\Pbar}{\overline{P}}
\newcommand{\Qbar}{\overline{Q}}
\newcommand{\Rbar}{\overline{R}}
\newcommand{\Sbar}{\overline{S}}
\newcommand{\Tbar}{\overline{T}}
\newcommand{\Ubar}{\overline{U}}
\newcommand{\Vbar}{\overline{V}}
\newcommand{\Wbar}{\overline{W}}
\newcommand{\Xbar}{\overline{X}}
\newcommand{\Ybar}{\overline{Y}}
\newcommand{\Zbar}{\overline{Z}}
\newcommand{\abar}{\overline{a}}
\newcommand{\bbar}{\overline{b}}
\newcommand{\cbar}{\overline{c}}
\newcommand{\dbar}{\overline{d}}
\newcommand{\ebar}{\overline{e}}
\newcommand{\fbar}{\overline{f}}
\newcommand{\gbar}{\overline{g}}
\renewcommand{\hbar}{\overline{h}}
\newcommand{\ibar}{\overline{i}}
\newcommand{\jbar}{\overline{j}}
\newcommand{\kbar}{\overline{k}}
\newcommand{\lbar}{\overline{l}}
\newcommand{\mbar}{\overline{m}}
\newcommand{\nbar}{\overline{n}}
\newcommand{\obar}{\overline{o}}
\newcommand{\pbar}{\overline{p}}
\newcommand{\qbar}{\overline{q}}
\newcommand{\rbar}{\overline{r}}
\newcommand{\sbar}{\overline{s}}
\newcommand{\tbar}{\overline{t}}
\newcommand{\ubar}{\overline{u}}
\newcommand{\vbar}{\overline{v}}
\newcommand{\wbar}{\overline{w}}
\newcommand{\xbar}{\overline{x}}
\newcommand{\ybar}{\overline{y}}
\newcommand{\zbar}{\overline{z}}
\newcommand\bigzero{\makebox(0,0){\text{\huge0}}}
\newcommand{\limp}{\lim\limits_{\leftarrow}}
\newcommand{\limi}{\lim\limits_{\rightarrow}}


\DeclareMathOperator{\End}{\mathrm{End}}
\DeclareMathOperator{\Hom}{\mathrm{Hom}}
\DeclareMathOperator{\Vect}{\mathrm{Vect}}
\DeclareMathOperator{\Spec}{\mathrm{Spec}}
\DeclareMathOperator{\multideg}{\mathrm{multideg}}
\DeclareMathOperator{\LM}{\mathrm{LM}}
\DeclareMathOperator{\LT}{\mathrm{LT}}
\DeclareMathOperator{\LC}{\mathrm{LC}}
\DeclareMathOperator{\PPCM}{\mathrm{PPCM}}
\DeclareMathOperator{\PGCD}{\mathrm{PGCD}}
\DeclareMathOperator{\Syl}{\mathrm{Syl}}
\DeclareMathOperator{\Res}{\mathrm{Res}}
\DeclareMathOperator{\Com}{\mathrm{Com}}
\DeclareMathOperator{\GL}{\mathrm{GL}}
\DeclareMathOperator{\SL}{\mathrm{SL}}
\DeclareMathOperator{\SU}{\mathrm{SU}}
\DeclareMathOperator{\PGL}{\mathrm{PGL}}
\DeclareMathOperator{\PSL}{\mathrm{PSL}}
\DeclareMathOperator{\PSU}{\mathrm{PSU}}
\DeclareMathOperator{\SO}{\mathrm{SO}}
\DeclareMathOperator{\Sp}{\mathrm{Sp}}
\DeclareMathOperator{\Spin}{\mathrm{Spin}}
\DeclareMathOperator{\Ker}{\mathrm{Ker}}
%\DeclareMathOperator{\Im}{\mathrm{Im}}

\DeclareMathOperator{\Ens}{\mathbf{Ens}}
\DeclareMathOperator{\Top}{\mathbf{Top}}
\DeclareMathOperator{\Ann}{\mathbf{Ann}}
\DeclareMathOperator{\Gr}{\mathbf{Gr}}
\DeclareMathOperator{\Ab}{\mathbf{Ab}}
%\DeclareMathOperator{\Vect}{\mathbf{Vect}}
\DeclareMathOperator{\Mod}{\mathbf{Mod}}
\headheight=0mm
\topmargin=-20mm
\oddsidemargin=-1cm
\evensidemargin=-1cm
\textwidth=18cm
\textheight=25cm
\parindent=0mm
\newif\ifproof
\newcommand{\demo}[1]{\ifproof #1 \else \fi}
 %Instruction d'utilisation : 
%les preuves du texte sont, en principe, entre des balises \demo, en sus des \begin{proof} pour l'instant.
%Laisser le texte tel quel, fait qu'elles ne sont pas affich�es.
%Mettre \prooftrue fait que toutes les preuves jusqu'� un \prooffalse ou la fin du document. 


 \begin{document}
\newtheorem{Thm}{Th�or�me}[chapter]
\newtheorem{Prop}[Thm]{Proposition}
\newtheorem{Propte}[Thm]{Propri�t�}
\newtheorem{Lemme}[Thm]{Lemme}
\newtheorem{Cor}[Thm]{Corollaire}


\theoremstyle{definition}

\newtheorem{Ex}[Thm]{Exemple}
\newtheorem{Def}[Thm]{D�finition}
\newtheorem{Defpropte}[Thm]{D�finition et propri�t�}
\newtheorem{Defprop}[Thm]{D�finition et proposition}
\newtheorem{Defthm}[Thm]{Th�or�me et d�finition}
\newtheorem{Not}[Thm]{Notation}
\newtheorem{Conv}[Thm]{Convention}
\newtheorem{Cons}[Thm]{Construction}

\theoremstyle{remark}
\newtheorem{Rq}[Thm]{Remarque}
\newtheorem{Slog}[Thm]{Slogan}
\newtheorem{Exo}[Thm]{Exercice}
\fi
\section{Modifications}
\label{Modif}
Dans cette section, nous allons \'etendre la d\'efinition de fonctions m\'eromorphes dans le cas o\`u le domaine d'arriv\'e est un espace analytique r\'eduit quelconque. Pour cela, on est amen\'e \`a d\'efinir la notion de modification dont un exemple simple est l'\'eclatement de l'origine dans $\C^{n+1}$.
%\subsection{Premier exemple : \'eclatement de l'origine dans $\C^{n+1}$}
\begin{Ex}
On commence par regarder le fibr\'e en droites tautologiques $ \Ocal_{\P^n}(-1)$ sur $\P^n$ d\'efini par :
\[ \Ocal_{\P^n}(-1):= \{ (L,z) \in \P^n\times \C^{n+1} \mid z \in L \}
\]
La projection de $\Ocal_{\P^n}(-1)$ sur $\C^{n+1}$ est submersive en tout point en dehors de $\P^n\times\{0\}=:\Ecal$. \\
En effet, si on regarde dans les cartes (donn\'ees par :
    \[\varphi :(L,z) \in T \cap (U_i \times C^{n+1}) \mapsto (\pi_i(L),z_i)\in \C^n \times \C
\]
 o\`u $U_i=\{z_i \neq 0\} \subset \P^n$ et $\pi : U_i \simeq \C^n$)
de $\Ocal_{\P^n}(-1)$, on obtient une application $\widetilde{\pi}_i : \C^{n+1} \to \C^{n+1}$ d\'efinie par :
\[  \forall (x_1,\ldots,x_n,z) \in \C^{n+1}, \widetilde{\pi_i}(x_1,\ldots,x_n,z)=z(x_1,\ldots,1,x_n)
\]
Son jacobien est donn\'e par :
\[
 \det Jac(\widetilde{\pi})=\det\begin{pmatrix}
    z&&&&&&z \\
    0&\ddots&&&&&\vdots\\
    \vdots&&z&&&&\vdots\\
    \vdots&&&0&&&1\\
    \vdots&&&z&&&\vdots\\
    \vdots&&&&\ddots&&\vdots\\
    0&&&&&z&x_n\\
    \end{pmatrix}=z^n
  \]
  Ainsi, $\widetilde{\pi}$ est submersive en tout point o\`u $z \neq 0$ (et donc $\pi$ aussi).
  Par le th\'eor\`eme d'inversion locale, $\pi : \Ocal_{\P^n}(-1)\setminus \Ecal \to \C^{n+1} \setminus \{0\}$ est un isomorphisme. On appelle \'eclatement de l'origine dans $\C^{n+1}$ la vari\'et\'e $\Ocal_{\P^n}(-1)$ muni de sa projection sur $\C^{n+1}$.
\end{Ex}

\begin{Def}
  Soit $\tau : \widetilde{X} \to X$ une application holomorphe entre deux espaces analytiques complexes r\'eduits. On dira que l'application $\tau$ est une modification si :
  \begin{itemize}
  \item $\tau$ est propre
    \item Il existe un sous-ensemble analytique $T$ d'int\'erieur vide dans $X$ tel que l'image r\'eciproque $\tau^{-1}(T)$ soit d'int\'erieur vide dans $\widetilde{X}$ et tel que $\tau$ induise un isomorphisme de $\widetilde{X}\setminus \tau^{-1}(T)$ sur $X \setminus T$.
    \end{itemize}
On appelera centre de la modification $\tau$ l'intersection des sous-ensembles analytiques $T$ de $X$ qui v\'erifient la condition ci-dessus (i.e. le plus petit d'entre eux).
\end{Def}

On peut remarquer qu'une modification est n\'ecessairement surjective car son image est un espace analytique ferm\'e (car $\tau$ est propre) qui contient un ouvert dense, celle-ci est donc $X$ tout entier.

\begin{Def}
Soit $X,Y$ deux espaces complexes r\'eduits. Une fonction m\'eromorphe de $X$ dans $Y$ est la donn\'ee d'un sous-espace analytique r\'eduit $\Gamma$ de $X \times Y$ muni d'une modification propre $\Gamma \to X$ et une application holomorphe $\Gamma \to Y$. On la note $X \dashrightarrow Y$
\end{Def}
Par d\'efinition de modification, une application m\'eromorphe $ X \dashrightarrow Y$ induit une application holomorphe d'un ouvert de Zariski de $X$ vers $Y$. \\
Cette d\'efinition co\"incide avec la d\'efinition donn\'ee dans \ref{def_fct_merom} : 
\begin{Lemme}
Soit $\widetilde{f}$ une fonction m\'eromorphe (comme d\'efini dans \ref{def_fct_merom}) sur un espace analytique irr\'eductible $X$. Pour chaque $(H,f) \in \widetilde{f}$, l'adh\'erence dans $X \times \P^1$ du graphe :
	\[\Gamma_{(H,f)}=\{(x,[z_0 : z_1]) \in (X \setminus H) \times \P^1 \mid z_0f(x)=z_1\}
	\]
	est un sous-espace analytique irr\'eductible dans $M \times \P^1$, ind\'ependant du choix du repr\'esentant de $\widetilde{f}$. La projection $\overline{\Gamma_{(H,f)}} \to M$ est une modification propre. Ainsi, $\widetilde{f}$ est une application m\'eromorphe $M \dashrightarrow \C$

\end{Lemme}

\begin{proof}
Soit $x_0 \in H$. Il existe un voisinage $U$ de $x_0$ dans $X$ et des applications holomorphes $g,h : U \to \C$ tels que les z\'eros de $h$ soient contenus dans $H$ et tels que $f$ et $g/h$ co\"incide sur $U\setminus U \cap H$. Ainsi, par d\'efinition de $\Gamma_{(H,f)}$, 
	\[\Gamma_{(H,f)} \cap (U \times \P^1) \subset \{ (x,[z_0 : z_1]) \in U \times \P^1 \mid z_0g(x)=z_1 h(x) \}=:Z
\]
Et donc l'adh\'erence $\overline{\Gamma_{(H,f)}}\cap (U \times \P^1)$ est aussi inclus dans $Z$. Ainsi, $\overline{\Gamma_{(H,f)}}\cap (U \times \P^1)$ est \'egal \`a la composante irr\'eductible de $Z$ la contenant. Cela prouve que $\overline{\Gamma_{(H,f)}}$ est un ensemble analytique de $M \times \P^1$. De plus, il est ind\'ependant du choix de $(H,f)$ car deux repr\'esentants $(H,f)$ et $(H',f)$ sont \'equivalents si $f$ et $f'$ sont \'egaux sur $X \setminus \{H \cup H'\}$
\end{proof}
%\begin{Def}
  %Soit $\pi : X \to Y$ une application holomorphe propre entre deux espaces analytiques complexes r\'eduits. On dira que $\pi$ est un morphisme localement projectif si, tout point $x \in Y$ a un voisinage ouvert tel qu'il existe une application holomorphe $j : \pi^{-1}(U) \to \P^k$ telle que l'application
        %\[(\pi,j) : \pi^{-1}(U) \to U \times \P^k
    %\]
     %soit un plongement propre. \\
 %On dira qu'un morphisme est localement projectif est projectif si l'on peut choisir $U=Y$. \\
  %Une modification qui est un morphisme localement projectif sera appel\'e modification localement projective.
%\end{Def}
%\begin{Rq}
%Les fibres d'un morphisme localement projectif sont isomorphes \`a un espace analytique inclus dans un espace projectif. Ainsi, par le th\'eor\`eme de Chow\footnote{voir \cite{}}, ce sont des ensembles alg\'ebriques.
%\end{Rq}
%
%\begin{Lemme}
%Soit $\pi : X\to Y$ une application propre et finie entre deux espaces complexes r\'eduits. Alors, $\pi$ est un morphisme localement projectif.
%\end{Lemme}
%
%\begin{proof}
  %Soit $x \in M$. La fibre $\pi^{-1}(x)$ \'etant fini, il existe un voisinage $V$ de $\pi^{-1}(X)$ et un plongement localement ferm\'e $j : \Omega \to \P^n$ (on envoie une union disjointe finie d'espaces analytiques mod\`eles dans l'espace projectif). Soit $U$ un voisinage de $x$ tel que $\pi^{-1}(U) \subset \Omega$. L'application $(\pi,j) : \pi^{-1}(U) \to U\times \P^n$ est un plongement propre : \\
  %Soit $K$ un compact de $U \times \P^n$. Alors, si on note par $p_1 : U \times \P^n \to \P^n$, on a l'inclusion :
  %\[
    %(\pi,j)^{-1}(K) \subset \pi^{-1}(\pi(K))
    %\]
%Comme $\pi$ est propre et que $\pi$ est continue alors $\pi^{-1}(\pi(K))$ est compact, et puisque $ (\pi,j)^{-1}(K)$ est, de plus, ferm\'e alors $(\pi,j)^{-1}(K)$ est compact. On conclut par injectivit\'e de $(\pi,j)$.
%\end{proof}
%\begin{Lemme}
%Soient $X$ un espace analytique complexe normal et $\tau : \widetilde{X} \to X$ une modification propre et finie de $M$. Alors $\tau$ est un isomorphisme.
%\end{Lemme}
%
%\begin{proof}
  %Soit $S$ le centre de la modification $\tau$. Celle-ci induit donc un isomorphisme de $\widetilde{X} \setminus \tau^{-1}(S) \to X \setminus S$. \\
  %Soit $x \in S$ et $j : V \to B$ un plongement d'un voisinage ouvert $V$ de $\tau^{-1}(x)$ dans un polydisque ouvert relativement compact $B \subset \C^n$. \\
  %Soit $W$ un voisinage ouvert de $x$ dans $M$ tel que $\tau^{-1}(W) \subset V$. Alors chacun des composantes de l'application holomorphe $j \circ \tau^{-1}: W \setminus S \to B$ est born\'ee et donc cette application se prolonge sur tout $W$ puisque $X$ est normal. Comme $j$ est un prolongement ferm\'e, $\tau^{-1}$ se prolonge holomorphiquement sur tout $M$.
  %\end{proof}

Au vu des propri\'et\'es des espaces normaux, lorsqu'on a une espace $X$ qui ne l'est pas, on voudrait trouver un espace normal $\widetilde{X}$ sans perdre trop d'informations sur $X$ :  

\begin{Def}
  Soit $X$ un espace analytique complexe r\'eduit. Une application holomorphe $\nu:Y \to X$ est appel\'e normalisation de $M$ si :
  \begin{itemize}
  \item L'espace complexe r\'eduit $N$ est normal.
  \item L'application $\nu$ est une modification propre de $M$.
  \item Les fibres de $\nu$ sont finies.
  \end{itemize}

\end{Def}

\begin{Thm}
Tout espace complexe r\'eduit admet une normalisation.
\end{Thm}
\begin{proof}
Voir \cite{\scv}



  \end{proof}









\ifwhole
 \end{document}
\fi



