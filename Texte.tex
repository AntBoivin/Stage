
\documentclass[a4page,10pt]{article}
\usepackage[Latin1]{inputenc}
\usepackage[francais]{babel}
\usepackage{amsmath,amssymb,amsthm}
\usepackage{textcomp}
\usepackage{mathrsfs}
\usepackage{algcompatible,algorithm  }
\usepackage[all]{xy}
\usepackage{hyperref}
\usepackage{fancyhdr}
\usepackage{supertabular}
\usepackage{makeidx}
\usepackage{setspace}
\usepackage{makeidx}
\usepackage[Bjornstrup]{fncychap}
\usepackage[nottoc,notlot,notlof]{tocbibind}
\makeindex
\pagestyle{fancy}
\fancyhead[L]{\leftmark}
\fancyhead[R]{}
\onehalfspacing

\toks0=\expandafter{\xy}
\edef\xy{\noexpand\shorthandoff{!?;:}\the\toks0 }
\makeatletter

\renewcommand*{\ALG@name}{Algorithme}

\makeatother

\renewcommand{\algorithmicrequire}{\textbf{\textsc {Entr\'ees  :  } } }
\renewcommand{\algorithmicensure}{\textbf{\textsc { Sortie  :  } } }
\renewcommand{\algorithmicwhile}{\textbf{Tant que}}
\renewcommand{\algorithmicdo}{\textbf{faire }}
\renewcommand{\algorithmicif}{\textbf{Si}}
\renewcommand{\algorithmicelse}{\textbf{Sinon}}
\renewcommand{\algorithmicthen}{\textbf{alors }}
\renewcommand{\algorithmicend}{\textbf{fin}}
\renewcommand{\algorithmicfor}{\textbf{Pour}}
\renewcommand{\algorithmicuntil}{\textbf{Jusqu'\`a}}
\renewcommand{\algorithmicrepeat}{\textbf{Répéter}}

\newcommand{\A}{\mathbb{A}}
\newcommand{\B}{\mathbb{B}}
\newcommand{\C}{\mathbb{C}}
\newcommand{\D}{\mathbb{D}}
\newcommand{\E}{\mathbb{E}}
\newcommand{\F}{\mathbb{F}}
\newcommand{\G}{\mathbb{G}}
\renewcommand{\H}{\mathbb{H}}
\newcommand{\I}{\mathbb{I}}
\newcommand{\J}{\mathbb{J}}
\newcommand{\K}{\mathbb{K}}
\renewcommand{\L}{\mathbb{L}}
\newcommand{\M}{\mathbb{M}}
\newcommand{\N}{\mathbb{N}}
\renewcommand{\O}{\mathbb{O}}
\renewcommand{\P}{\mathbb{P}}
\newcommand{\Q}{\mathbb{Q}}
\newcommand{\R}{\mathbb{R}}
\renewcommand{\S}{\mathbb{S}}
\newcommand{\T}{\mathbb{T}}
\newcommand{\U}{\mathbb{U}}
\newcommand{\V}{\mathbb{V}}
\newcommand{\W}{\mathbb{W}}
\newcommand{\X}{\mathbb{X}}
\newcommand{\Y}{\mathbb{Y}}
\newcommand{\Z}{\mathbb{Z}}
\newcommand{\Acal}{\mathcal{A}}
\newcommand{\Bcal}{\mathcal{B}}
\newcommand{\Ccal}{\mathcal{C}}
\newcommand{\Dcal}{\mathcal{D}}
\newcommand{\Ecal}{\mathcal{E}}
\newcommand{\Fcal}{\mathcal{F}}
\newcommand{\Gcal}{\mathcal{G}}
\newcommand{\Hcal}{\mathcal{H}}
\newcommand{\Ical}{\mathcal{I}}
\newcommand{\Jcal}{\mathcal{J}}
\newcommand{\Kcal}{\mathcal{K}}
\newcommand{\Lcal}{\mathcal{L}}
\newcommand{\Mcal}{\mathcal{M}}
\newcommand{\Ncal}{\mathcal{N}}
\newcommand{\Ocal}{\mathcal{O}}
\newcommand{\Pcal}{\mathcal{P}}
\newcommand{\Qcal}{\mathcal{Q}}
\newcommand{\Rcal}{\mathcal{R}}
\newcommand{\Scal}{\mathcal{S}}
\newcommand{\Tcal}{\mathcal{T}}
\newcommand{\Ucal}{\mathcal{U}}
\newcommand{\Vcal}{\mathcal{V}}
\newcommand{\Wcal}{\mathcal{W}}
\newcommand{\Xcal}{\mathcal{X}}
\newcommand{\Ycal}{\mathcal{Y}}
\newcommand{\Zcal}{\mathcal{Z}}
\newcommand{\Ascr}{\mathscr{A}}
\newcommand{\Bscr}{\mathscr{B}}
\newcommand{\Cscr}{\mathscr{C}}
\newcommand{\Dscr}{\mathscr{D}}
\newcommand{\Escr}{\mathscr{E}}
\newcommand{\Fscr}{\mathscr{F}}
\newcommand{\Gscr}{\mathscr{G}}
\newcommand{\Hscr}{\mathscr{H}}
\newcommand{\Iscr}{\mathscr{I}}
\newcommand{\Jscr}{\mathscr{J}}
\newcommand{\Kscr}{\mathscr{K}}
\newcommand{\Lscr}{\mathscr{L}}
\newcommand{\Mscr}{\mathscr{M}}
\newcommand{\Nscr}{\mathscr{N}}
\newcommand{\Oscr}{\mathscr{O}}
\newcommand{\Pscr}{\mathscr{P}}
\newcommand{\Qscr}{\mathscr{Q}}
\newcommand{\Rscr}{\mathscr{R}}
\newcommand{\Sscr}{\mathscr{S}}
\newcommand{\Tscr}{\mathscr{T}}
\newcommand{\Uscr}{\mathscr{U}}
\newcommand{\Vscr}{\mathscr{V}}
\newcommand{\Wscr}{\mathscr{W}}
\newcommand{\Xscr}{\mathscr{X}}
\newcommand{\Yscr}{\mathscr{Y}}
\newcommand{\Zscr}{\mathscr{Z}}
\newcommand{\Afrak}{\mathfrak{A}}
\newcommand{\Bfrak}{\mathfrak{B}}
\newcommand{\Cfrak}{\mathfrak{C}}
\newcommand{\Dfrak}{\mathfrak{D}}
\newcommand{\Efrak}{\mathfrak{E}}
\newcommand{\Ffrak}{\mathfrak{F}}
\newcommand{\Gfrak}{\mathfrak{G}}
\newcommand{\Hfrak}{\mathfrak{H}}
\newcommand{\Ifrak}{\mathfrak{I}}
\newcommand{\Jfrak}{\mathfrak{J}}
\newcommand{\Kfrak}{\mathfrak{K}}
\newcommand{\Lfrak}{\mathfrak{L}}
\newcommand{\Mfrak}{\mathfrak{M}}
\newcommand{\Nfrak}{\mathfrak{N}}
\newcommand{\Ofrak}{\mathfrak{O}}
\newcommand{\Pfrak}{\mathfrak{P}}
\newcommand{\Qfrak}{\mathfrak{Q}}
\newcommand{\Rfrak}{\mathfrak{R}}
\newcommand{\Sfrak}{\mathfrak{S}}
\newcommand{\Tfrak}{\mathfrak{T}}
\newcommand{\Ufrak}{\mathfrak{U}}
\newcommand{\Vfrak}{\mathfrak{V}}
\newcommand{\Wfrak}{\mathfrak{W}}
\newcommand{\Xfrak}{\mathfrak{X}}
\newcommand{\Yfrak}{\mathfrak{Y}}
\newcommand{\Zfrak}{\mathfrak{Z}}
\newcommand{\afrak}{\mathfrak{a}}
\newcommand{\bfrak}{\mathfrak{b}}
\newcommand{\cfrak}{\mathfrak{c}}
\newcommand{\dfrak}{\mathfrak{d}}
\newcommand{\efrak}{\mathfrak{e}}
\newcommand{\ffrak}{\mathfrak{f}}
\newcommand{\gfrak}{\mathfrak{g}}
\newcommand{\hfrak}{\mathfrak{h}}
\newcommand{\ifrak}{\mathfrak{i}}
\newcommand{\jfrak}{\mathfrak{j}}
\newcommand{\kfrak}{\mathfrak{k}}
\newcommand{\lfrak}{\mathfrak{l}}
\newcommand{\mfrak}{\mathfrak{m}}
\newcommand{\nfrak}{\mathfrak{n}}
\newcommand{\ofrak}{\mathfrak{o}}
\newcommand{\pfrak}{\mathfrak{p}}
\newcommand{\qfrak}{\mathfrak{q}}
\newcommand{\rfrak}{\mathfrak{r}}
\newcommand{\sfrak}{\mathfrak{s}}
\newcommand{\tfrak}{\mathfrak{t}}
\newcommand{\ufrak}{\mathfrak{u}}
\newcommand{\vfrak}{\mathfrak{v}}
\newcommand{\wfrak}{\mathfrak{w}}
\newcommand{\xfrak}{\mathfrak{x}}
\newcommand{\yfrak}{\mathfrak{y}}
\newcommand{\zfrak}{\mathfrak{z}}
\newcommand{\Abar}{\overline{A}}
\newcommand{\Bbar}{\overline{B}}
\newcommand{\Cbar}{\overline{C}}
\newcommand{\Dbar}{\overline{D}}
\newcommand{\Ebar}{\overline{E}}
\newcommand{\Fbar}{\overline{F}}
\newcommand{\Gbar}{\overline{G}}
\newcommand{\Hbar}{\overline{H}}
\newcommand{\Ibar}{\overline{I}}
\newcommand{\Jbar}{\overline{J}}
\newcommand{\Kbar}{\overline{K}}
\newcommand{\Lbar}{\overline{L}}
\newcommand{\Mbar}{\overline{M}}
\newcommand{\Nbar}{\overline{N}}
\newcommand{\Obar}{\overline{O}}
\newcommand{\Pbar}{\overline{P}}
\newcommand{\Qbar}{\overline{Q}}
\newcommand{\Rbar}{\overline{R}}
\newcommand{\Sbar}{\overline{S}}
\newcommand{\Tbar}{\overline{T}}
\newcommand{\Ubar}{\overline{U}}
\newcommand{\Vbar}{\overline{V}}
\newcommand{\Wbar}{\overline{W}}
\newcommand{\Xbar}{\overline{X}}
\newcommand{\Ybar}{\overline{Y}}
\newcommand{\Zbar}{\overline{Z}}
\newcommand{\abar}{\overline{a}}
\newcommand{\bbar}{\overline{b}}
\newcommand{\cbar}{\overline{c}}
\newcommand{\dbar}{\overline{d}}
\newcommand{\ebar}{\overline{e}}
\newcommand{\fbar}{\overline{f}}
\newcommand{\gbar}{\overline{g}}
\renewcommand{\hbar}{\overline{h}}
\newcommand{\ibar}{\overline{i}}
\newcommand{\jbar}{\overline{j}}
\newcommand{\kbar}{\overline{k}}
\newcommand{\lbar}{\overline{l}}
\newcommand{\mbar}{\overline{m}}
\newcommand{\nbar}{\overline{n}}
\newcommand{\obar}{\overline{o}}
\newcommand{\pbar}{\overline{p}}
\newcommand{\qbar}{\overline{q}}
\newcommand{\rbar}{\overline{r}}
\newcommand{\sbar}{\overline{s}}
\newcommand{\tbar}{\overline{t}}
\newcommand{\ubar}{\overline{u}}
\newcommand{\vbar}{\overline{v}}
\newcommand{\wbar}{\overline{w}}
\newcommand{\xbar}{\overline{x}}
\newcommand{\ybar}{\overline{y}}
\newcommand{\zbar}{\overline{z}}
\newcommand\bigzero{\makebox(0,0){\text{\huge0}}}
\newcommand{\limp}{\lim\limits_{\leftarrow}}
\newcommand{\limi}{\lim\limits_{\rightarrow}}


\DeclareMathOperator{\End}{\mathrm{End}}
\DeclareMathOperator{\Hom}{\mathrm{Hom}}
\DeclareMathOperator{\Vect}{\mathrm{Vect}}
\DeclareMathOperator{\Spec}{\mathrm{Spec}}
\DeclareMathOperator{\multideg}{\mathrm{multideg}}
\DeclareMathOperator{\LM}{\mathrm{LM}}
\DeclareMathOperator{\LT}{\mathrm{LT}}
\DeclareMathOperator{\LC}{\mathrm{LC}}
\DeclareMathOperator{\PPCM}{\mathrm{PPCM}}
\DeclareMathOperator{\PGCD}{\mathrm{PGCD}}
\DeclareMathOperator{\Syl}{\mathrm{Syl}}
\DeclareMathOperator{\Res}{\mathrm{Res}}
\DeclareMathOperator{\Com}{\mathrm{Com}}
\DeclareMathOperator{\GL}{\mathrm{GL}}
\DeclareMathOperator{\SL}{\mathrm{SL}}
\DeclareMathOperator{\SU}{\mathrm{SU}}
\DeclareMathOperator{\PGL}{\mathrm{PGL}}
\DeclareMathOperator{\PSL}{\mathrm{PSL}}
\DeclareMathOperator{\PSU}{\mathrm{PSU}}
\DeclareMathOperator{\SO}{\mathrm{SO}}
\DeclareMathOperator{\Sp}{\mathrm{Sp}}
\DeclareMathOperator{\Spin}{\mathrm{Spin}}
\DeclareMathOperator{\Ker}{\mathrm{Ker}}
%\DeclareMathOperator{\Im}{\mathrm{Im}}

\DeclareMathOperator{\Ens}{\mathbf{Ens}}
\DeclareMathOperator{\Top}{\mathbf{Top}}
\DeclareMathOperator{\Ann}{\mathbf{Ann}}
\DeclareMathOperator{\Gr}{\mathbf{Gr}}
\DeclareMathOperator{\Ab}{\mathbf{Ab}}
%\DeclareMathOperator{\Vect}{\mathbf{Vect}}
\DeclareMathOperator{\Mod}{\mathbf{Mod}}
\headheight=0mm
\topmargin=-20mm
\oddsidemargin=-1cm
\evensidemargin=-1cm
\textwidth=18cm
\textheight=25cm
\parindent=0mm
\newif\ifproof
\newcommand{\demo}[1]{\ifproof #1 \else \fi}
 %Instruction d'utilisation : 
%les preuves du texte sont, en principe, entre des balises \demo, en sus des \begin{proof} pour l'instant.
%Laisser le texte tel quel, fait qu'elles ne sont pas affich�es.
%Mettre \prooftrue fait que toutes les preuves jusqu'� un \prooffalse ou la fin du document. 


\begin{document}
\newtheorem{Thm}{Th�or�me}[chapter]
\newtheorem{Prop}[Thm]{Proposition}
\newtheorem{Propte}[Thm]{Propri�t�}
\newtheorem{Lemme}[Thm]{Lemme}
\newtheorem{Cor}[Thm]{Corollaire}


\theoremstyle{definition}

\newtheorem{Ex}[Thm]{Exemple}
\newtheorem{Def}[Thm]{D�finition}
\newtheorem{Defpropte}[Thm]{D�finition et propri�t�}
\newtheorem{Defprop}[Thm]{D�finition et proposition}
\newtheorem{Defthm}[Thm]{Th�or�me et d�finition}
\newtheorem{Not}[Thm]{Notation}
\newtheorem{Conv}[Thm]{Convention}
\newtheorem{Cons}[Thm]{Construction}

\theoremstyle{remark}
\newtheorem{Rq}[Thm]{Remarque}
\newtheorem{Slog}[Thm]{Slogan}
\newtheorem{Exo}[Thm]{Exercice}




\newcommand{\cac}{barlet_cycles_2014}
\newcommand{\btg}{bourbaki_topologie_2007}   
\newcommand{\pag}{griffiths_principles_2011}
\newcommand{\barlet}{barlet_espace_1975}
\newcommand{\voisin}{voisin_theorie_2002}
\newcommand{\scv}{grauert_several_2013}


\newif\ifwhole
%\wholetrue
% Ajouter \wholetrue si on compile seulement ce fichier

\ifwhole
 \documentclass[a4page,10pt]{article}
     \usepackage[Latin1]{inputenc}
\usepackage[francais]{babel}
\usepackage{amsmath,amssymb,amsthm}
\usepackage{textcomp}
\usepackage{mathrsfs}
\usepackage{algcompatible,algorithm  }
\usepackage[all]{xy}
\usepackage{hyperref}
\usepackage{fancyhdr}
\usepackage{supertabular}
\usepackage{makeidx}
\usepackage{setspace}
\usepackage{makeidx}
\usepackage[Bjornstrup]{fncychap}
\usepackage[nottoc,notlot,notlof]{tocbibind}
\makeindex
\pagestyle{fancy}
\fancyhead[L]{\leftmark}
\fancyhead[R]{}
\onehalfspacing

\toks0=\expandafter{\xy}
\edef\xy{\noexpand\shorthandoff{!?;:}\the\toks0 }
\makeatletter

\renewcommand*{\ALG@name}{Algorithme}

\makeatother

\renewcommand{\algorithmicrequire}{\textbf{\textsc {Entr\'ees  :  } } }
\renewcommand{\algorithmicensure}{\textbf{\textsc { Sortie  :  } } }
\renewcommand{\algorithmicwhile}{\textbf{Tant que}}
\renewcommand{\algorithmicdo}{\textbf{faire }}
\renewcommand{\algorithmicif}{\textbf{Si}}
\renewcommand{\algorithmicelse}{\textbf{Sinon}}
\renewcommand{\algorithmicthen}{\textbf{alors }}
\renewcommand{\algorithmicend}{\textbf{fin}}
\renewcommand{\algorithmicfor}{\textbf{Pour}}
\renewcommand{\algorithmicuntil}{\textbf{Jusqu'\`a}}
\renewcommand{\algorithmicrepeat}{\textbf{Répéter}}

\newcommand{\A}{\mathbb{A}}
\newcommand{\B}{\mathbb{B}}
\newcommand{\C}{\mathbb{C}}
\newcommand{\D}{\mathbb{D}}
\newcommand{\E}{\mathbb{E}}
\newcommand{\F}{\mathbb{F}}
\newcommand{\G}{\mathbb{G}}
\renewcommand{\H}{\mathbb{H}}
\newcommand{\I}{\mathbb{I}}
\newcommand{\J}{\mathbb{J}}
\newcommand{\K}{\mathbb{K}}
\renewcommand{\L}{\mathbb{L}}
\newcommand{\M}{\mathbb{M}}
\newcommand{\N}{\mathbb{N}}
\renewcommand{\O}{\mathbb{O}}
\renewcommand{\P}{\mathbb{P}}
\newcommand{\Q}{\mathbb{Q}}
\newcommand{\R}{\mathbb{R}}
\renewcommand{\S}{\mathbb{S}}
\newcommand{\T}{\mathbb{T}}
\newcommand{\U}{\mathbb{U}}
\newcommand{\V}{\mathbb{V}}
\newcommand{\W}{\mathbb{W}}
\newcommand{\X}{\mathbb{X}}
\newcommand{\Y}{\mathbb{Y}}
\newcommand{\Z}{\mathbb{Z}}
\newcommand{\Acal}{\mathcal{A}}
\newcommand{\Bcal}{\mathcal{B}}
\newcommand{\Ccal}{\mathcal{C}}
\newcommand{\Dcal}{\mathcal{D}}
\newcommand{\Ecal}{\mathcal{E}}
\newcommand{\Fcal}{\mathcal{F}}
\newcommand{\Gcal}{\mathcal{G}}
\newcommand{\Hcal}{\mathcal{H}}
\newcommand{\Ical}{\mathcal{I}}
\newcommand{\Jcal}{\mathcal{J}}
\newcommand{\Kcal}{\mathcal{K}}
\newcommand{\Lcal}{\mathcal{L}}
\newcommand{\Mcal}{\mathcal{M}}
\newcommand{\Ncal}{\mathcal{N}}
\newcommand{\Ocal}{\mathcal{O}}
\newcommand{\Pcal}{\mathcal{P}}
\newcommand{\Qcal}{\mathcal{Q}}
\newcommand{\Rcal}{\mathcal{R}}
\newcommand{\Scal}{\mathcal{S}}
\newcommand{\Tcal}{\mathcal{T}}
\newcommand{\Ucal}{\mathcal{U}}
\newcommand{\Vcal}{\mathcal{V}}
\newcommand{\Wcal}{\mathcal{W}}
\newcommand{\Xcal}{\mathcal{X}}
\newcommand{\Ycal}{\mathcal{Y}}
\newcommand{\Zcal}{\mathcal{Z}}
\newcommand{\Ascr}{\mathscr{A}}
\newcommand{\Bscr}{\mathscr{B}}
\newcommand{\Cscr}{\mathscr{C}}
\newcommand{\Dscr}{\mathscr{D}}
\newcommand{\Escr}{\mathscr{E}}
\newcommand{\Fscr}{\mathscr{F}}
\newcommand{\Gscr}{\mathscr{G}}
\newcommand{\Hscr}{\mathscr{H}}
\newcommand{\Iscr}{\mathscr{I}}
\newcommand{\Jscr}{\mathscr{J}}
\newcommand{\Kscr}{\mathscr{K}}
\newcommand{\Lscr}{\mathscr{L}}
\newcommand{\Mscr}{\mathscr{M}}
\newcommand{\Nscr}{\mathscr{N}}
\newcommand{\Oscr}{\mathscr{O}}
\newcommand{\Pscr}{\mathscr{P}}
\newcommand{\Qscr}{\mathscr{Q}}
\newcommand{\Rscr}{\mathscr{R}}
\newcommand{\Sscr}{\mathscr{S}}
\newcommand{\Tscr}{\mathscr{T}}
\newcommand{\Uscr}{\mathscr{U}}
\newcommand{\Vscr}{\mathscr{V}}
\newcommand{\Wscr}{\mathscr{W}}
\newcommand{\Xscr}{\mathscr{X}}
\newcommand{\Yscr}{\mathscr{Y}}
\newcommand{\Zscr}{\mathscr{Z}}
\newcommand{\Afrak}{\mathfrak{A}}
\newcommand{\Bfrak}{\mathfrak{B}}
\newcommand{\Cfrak}{\mathfrak{C}}
\newcommand{\Dfrak}{\mathfrak{D}}
\newcommand{\Efrak}{\mathfrak{E}}
\newcommand{\Ffrak}{\mathfrak{F}}
\newcommand{\Gfrak}{\mathfrak{G}}
\newcommand{\Hfrak}{\mathfrak{H}}
\newcommand{\Ifrak}{\mathfrak{I}}
\newcommand{\Jfrak}{\mathfrak{J}}
\newcommand{\Kfrak}{\mathfrak{K}}
\newcommand{\Lfrak}{\mathfrak{L}}
\newcommand{\Mfrak}{\mathfrak{M}}
\newcommand{\Nfrak}{\mathfrak{N}}
\newcommand{\Ofrak}{\mathfrak{O}}
\newcommand{\Pfrak}{\mathfrak{P}}
\newcommand{\Qfrak}{\mathfrak{Q}}
\newcommand{\Rfrak}{\mathfrak{R}}
\newcommand{\Sfrak}{\mathfrak{S}}
\newcommand{\Tfrak}{\mathfrak{T}}
\newcommand{\Ufrak}{\mathfrak{U}}
\newcommand{\Vfrak}{\mathfrak{V}}
\newcommand{\Wfrak}{\mathfrak{W}}
\newcommand{\Xfrak}{\mathfrak{X}}
\newcommand{\Yfrak}{\mathfrak{Y}}
\newcommand{\Zfrak}{\mathfrak{Z}}
\newcommand{\afrak}{\mathfrak{a}}
\newcommand{\bfrak}{\mathfrak{b}}
\newcommand{\cfrak}{\mathfrak{c}}
\newcommand{\dfrak}{\mathfrak{d}}
\newcommand{\efrak}{\mathfrak{e}}
\newcommand{\ffrak}{\mathfrak{f}}
\newcommand{\gfrak}{\mathfrak{g}}
\newcommand{\hfrak}{\mathfrak{h}}
\newcommand{\ifrak}{\mathfrak{i}}
\newcommand{\jfrak}{\mathfrak{j}}
\newcommand{\kfrak}{\mathfrak{k}}
\newcommand{\lfrak}{\mathfrak{l}}
\newcommand{\mfrak}{\mathfrak{m}}
\newcommand{\nfrak}{\mathfrak{n}}
\newcommand{\ofrak}{\mathfrak{o}}
\newcommand{\pfrak}{\mathfrak{p}}
\newcommand{\qfrak}{\mathfrak{q}}
\newcommand{\rfrak}{\mathfrak{r}}
\newcommand{\sfrak}{\mathfrak{s}}
\newcommand{\tfrak}{\mathfrak{t}}
\newcommand{\ufrak}{\mathfrak{u}}
\newcommand{\vfrak}{\mathfrak{v}}
\newcommand{\wfrak}{\mathfrak{w}}
\newcommand{\xfrak}{\mathfrak{x}}
\newcommand{\yfrak}{\mathfrak{y}}
\newcommand{\zfrak}{\mathfrak{z}}
\newcommand{\Abar}{\overline{A}}
\newcommand{\Bbar}{\overline{B}}
\newcommand{\Cbar}{\overline{C}}
\newcommand{\Dbar}{\overline{D}}
\newcommand{\Ebar}{\overline{E}}
\newcommand{\Fbar}{\overline{F}}
\newcommand{\Gbar}{\overline{G}}
\newcommand{\Hbar}{\overline{H}}
\newcommand{\Ibar}{\overline{I}}
\newcommand{\Jbar}{\overline{J}}
\newcommand{\Kbar}{\overline{K}}
\newcommand{\Lbar}{\overline{L}}
\newcommand{\Mbar}{\overline{M}}
\newcommand{\Nbar}{\overline{N}}
\newcommand{\Obar}{\overline{O}}
\newcommand{\Pbar}{\overline{P}}
\newcommand{\Qbar}{\overline{Q}}
\newcommand{\Rbar}{\overline{R}}
\newcommand{\Sbar}{\overline{S}}
\newcommand{\Tbar}{\overline{T}}
\newcommand{\Ubar}{\overline{U}}
\newcommand{\Vbar}{\overline{V}}
\newcommand{\Wbar}{\overline{W}}
\newcommand{\Xbar}{\overline{X}}
\newcommand{\Ybar}{\overline{Y}}
\newcommand{\Zbar}{\overline{Z}}
\newcommand{\abar}{\overline{a}}
\newcommand{\bbar}{\overline{b}}
\newcommand{\cbar}{\overline{c}}
\newcommand{\dbar}{\overline{d}}
\newcommand{\ebar}{\overline{e}}
\newcommand{\fbar}{\overline{f}}
\newcommand{\gbar}{\overline{g}}
\renewcommand{\hbar}{\overline{h}}
\newcommand{\ibar}{\overline{i}}
\newcommand{\jbar}{\overline{j}}
\newcommand{\kbar}{\overline{k}}
\newcommand{\lbar}{\overline{l}}
\newcommand{\mbar}{\overline{m}}
\newcommand{\nbar}{\overline{n}}
\newcommand{\obar}{\overline{o}}
\newcommand{\pbar}{\overline{p}}
\newcommand{\qbar}{\overline{q}}
\newcommand{\rbar}{\overline{r}}
\newcommand{\sbar}{\overline{s}}
\newcommand{\tbar}{\overline{t}}
\newcommand{\ubar}{\overline{u}}
\newcommand{\vbar}{\overline{v}}
\newcommand{\wbar}{\overline{w}}
\newcommand{\xbar}{\overline{x}}
\newcommand{\ybar}{\overline{y}}
\newcommand{\zbar}{\overline{z}}
\newcommand\bigzero{\makebox(0,0){\text{\huge0}}}
\newcommand{\limp}{\lim\limits_{\leftarrow}}
\newcommand{\limi}{\lim\limits_{\rightarrow}}


\DeclareMathOperator{\End}{\mathrm{End}}
\DeclareMathOperator{\Hom}{\mathrm{Hom}}
\DeclareMathOperator{\Vect}{\mathrm{Vect}}
\DeclareMathOperator{\Spec}{\mathrm{Spec}}
\DeclareMathOperator{\multideg}{\mathrm{multideg}}
\DeclareMathOperator{\LM}{\mathrm{LM}}
\DeclareMathOperator{\LT}{\mathrm{LT}}
\DeclareMathOperator{\LC}{\mathrm{LC}}
\DeclareMathOperator{\PPCM}{\mathrm{PPCM}}
\DeclareMathOperator{\PGCD}{\mathrm{PGCD}}
\DeclareMathOperator{\Syl}{\mathrm{Syl}}
\DeclareMathOperator{\Res}{\mathrm{Res}}
\DeclareMathOperator{\Com}{\mathrm{Com}}
\DeclareMathOperator{\GL}{\mathrm{GL}}
\DeclareMathOperator{\SL}{\mathrm{SL}}
\DeclareMathOperator{\SU}{\mathrm{SU}}
\DeclareMathOperator{\PGL}{\mathrm{PGL}}
\DeclareMathOperator{\PSL}{\mathrm{PSL}}
\DeclareMathOperator{\PSU}{\mathrm{PSU}}
\DeclareMathOperator{\SO}{\mathrm{SO}}
\DeclareMathOperator{\Sp}{\mathrm{Sp}}
\DeclareMathOperator{\Spin}{\mathrm{Spin}}
\DeclareMathOperator{\Ker}{\mathrm{Ker}}
%\DeclareMathOperator{\Im}{\mathrm{Im}}

\DeclareMathOperator{\Ens}{\mathbf{Ens}}
\DeclareMathOperator{\Top}{\mathbf{Top}}
\DeclareMathOperator{\Ann}{\mathbf{Ann}}
\DeclareMathOperator{\Gr}{\mathbf{Gr}}
\DeclareMathOperator{\Ab}{\mathbf{Ab}}
%\DeclareMathOperator{\Vect}{\mathbf{Vect}}
\DeclareMathOperator{\Mod}{\mathbf{Mod}}
\headheight=0mm
\topmargin=-20mm
\oddsidemargin=-1cm
\evensidemargin=-1cm
\textwidth=18cm
\textheight=25cm
\parindent=0mm
\newif\ifproof
\newcommand{\demo}[1]{\ifproof #1 \else \fi}
 %Instruction d'utilisation : 
%les preuves du texte sont, en principe, entre des balises \demo, en sus des \begin{proof} pour l'instant.
%Laisser le texte tel quel, fait qu'elles ne sont pas affich�es.
%Mettre \prooftrue fait que toutes les preuves jusqu'� un \prooffalse ou la fin du document. 


 \begin{document}
\newtheorem{Thm}{Th�or�me}[chapter]
\newtheorem{Prop}[Thm]{Proposition}
\newtheorem{Propte}[Thm]{Propri�t�}
\newtheorem{Lemme}[Thm]{Lemme}
\newtheorem{Cor}[Thm]{Corollaire}


\theoremstyle{definition}

\newtheorem{Ex}[Thm]{Exemple}
\newtheorem{Def}[Thm]{D�finition}
\newtheorem{Defpropte}[Thm]{D�finition et propri�t�}
\newtheorem{Defprop}[Thm]{D�finition et proposition}
\newtheorem{Defthm}[Thm]{Th�or�me et d�finition}
\newtheorem{Not}[Thm]{Notation}
\newtheorem{Conv}[Thm]{Convention}
\newtheorem{Cons}[Thm]{Construction}

\theoremstyle{remark}
\newtheorem{Rq}[Thm]{Remarque}
\newtheorem{Slog}[Thm]{Slogan}
\newtheorem{Exo}[Thm]{Exercice}

\fi

\section{$Sym^k(\C^p)$}

\begin{Def}
Soit $M$ un espace topologique et $k \in \N^*$. On appelle $k$�me produit sym�trique de $M$ l'espace topologique $Sym^k(M)$, quotient  de $M^k$ par l'action de $\Sfrak_k$ donn�e par, pour tout $\sigma \in \Sfrak_k$ et pour tout $(z_1,\ldots,z_k) \in M^k$, 
	\[\sigma\cdot (z_1,\ldots,z_n)=(z_{\sigma(1)},\ldots,z_{\sigma(k)})
\]
\end{Def}

\begin{Prop}
Si $M$ est s�par� alors $Sym^k(M)$ l'est aussi. De plus, si $M$ est compact, localement compact ou d�nombrable � l'infini, $Sym^k(M)$ l'est aussi.
\end{Prop}
\begin{proof}
Supposons $M$ s�par�. \\
Soit $[x] \neq [y] \in Sym^k(M)$. \\
Soient $x_\sigma, \sigma \in \Sfrak_k$ les repr�sentants de $[x]$ et $y_\tau, \tau \in \Sfrak_k$ ceux de $[y]$. \\
Comme $M$ est s�par�, il existe, pour tout $\sigma,\tau \in \Sfrak_k$, un voisinage $V_{\sigma\tau}$ de $x_\sigma$  et un voisinage $W_{\tau\sigma }$ de $y_\tau$ tels que $V_{\sigma \tau} \cap W_{ \tau\sigma}=\emptyset$. On se fixe un $V_{\sigma\tau}$ et un $W_{ \tau\sigma}$ pour tout $\tau$ et tout $\sigma$.   \\
Soient $V_\sigma:=\bigcap_\tau V_{\sigma \tau}$ et $W_\tau:=\bigcap_\sigma W_{\tau\sigma }$. On remarque que, pour tout $\tau$ et tout $\sigma$, 
	\[V_\lambda \cap  W_\mu=\emptyset
\]
Soient $V:=\bigcap \sigma^{-1} V_\sigma$ et $W=\bigcap \tau^{-1} W_\tau$. Ce sont des voisinages de respectivement $x_{id}$ et $y_{id}$. 
Les ensembles $\pi(V)$ et $\pi(W)$ contiennent respectivement $[x]$ et $[y]$ et sont disjoints. $Sym^k(M)$ est donc s�par�.\\
En effet, supposons qu'il existe $[z] \in \pi(V) \cap \pi(W)$ alors, si on note $z$ un de ses repr�sentants, il existe $\lambda,\mu \in \Sfrak_k$ tels que : 
	\[ \lambda z \in V \text{ et } \mu z \in W
	\]
Ainsi, 
	\[ z \in \lambda^{-1} V \cap \mu^{-1} W \subset  V_\lambda \cap  W_\mu
\]
	 D'o�, la contradiction.

\end{proof}




Dans la suite, on consid�rera $M=\C^p$ (ou ses ouverts) avec la topologie classique. 

Pour donner une structure complexe sur $Sym^k(\C^p)$, nous allons montrer que $Sym^k(\C^p)$ est un sous-ensemble alg�brique de $\bigoplus_{j=0}^k S^j(\C^p)$ o� $S(\C^p)=\bigoplus_{ i \geq 0} S^i(\C^p)$ est l'alg�bre sym�trique de $\C^p$ : \\
Soit $S_j :  Sym^k(\C^p) \to S^j(\C^p)$ l'application d�finie pour tout $j \in \{1,\ldots,k\}$ par : 
	\[S_j(x_1,\ldots,x_k)=\sum_{1 \leq i_1 < \ldots < i_j \leq k} x_{i_1} \ldots x_{i_j}
\]
\begin{Prop} \label{plongement}
L'application $S=\bigoplus_{j=0}^k S_i : Sym^k(\C^p) \to \bigoplus_{j=0}^k S^j(\C^p)$ est propre et induit un hom�morphisme sur son image. Cette image est un sous-ensemble alg�brique de $\bigoplus_{j=0}^k S^j(\C^p)$
\end{Prop}
\begin{proof}
cf \cite{\cac} p 67
\end{proof}

\begin{Thm}
Soient $V$ un ouvert de $Sym^k(\C^p)$ et $M$ une vari�t� complexe. 
\begin{itemize}
	\item Une fonction $f : V \to \C$ est holomorphe si, et seulement si la fonction $f \circ \pi : \pi^{-1}(V) \to \C$ est une fonction holomorphe (o� $\pi : (\C^p)^k \to Sym^k(\C^p)$ est la projection canonique).
	\item Une application $M \to V$ est dite holomorphe si c'est une application continue telle que pour toute fonction holomorphe $Sym^k(\C^p)$, leur compos�e est holomorphe.
\end{itemize}
\end{Thm}
\begin{proof}
\cite{\cac} p 78-85
\end{proof}
\section{Rev�tements ramifi�s}
\begin{Def}
Soit $U$ un espace analytique normal de dimension finie. Soit $B$ un polydisque de $\C^p$ (qui peut �tre �gal � $\C^p$). \\
On appelera rev�tement ramifi� de degr� $k$ de $U$ , contenu dans $U \times B$, la donn�e d'un nombre fini de sous-ensembles analytiques irr�ductibles $X_i$ ferm�s disjoints dans $U \times B$, affect�s de multiplicit�s $n_i > 0$, tels que la restriction � chaque $X_i$ de la projection naturelle $U \times B \to U$ , soit propre, surjective et de degr� $k_i$ \footnote{ici, propre implique finie}, de sorte que l'on
ait $\sum n_ik_i=k$. \\
On notera $|X|$ le rev�tement ramifi� $(X_i,1)$ que l'on appelera support de $X$ qu'on identifiera avec l'ensemble analytique $\bigcup X_i$
\end{Def}

\begin{Prop}\label{bij_rr_appli}

Soit $U$ un espace analytique normal de dimension finie et soit $B$ un polydisque de $\C^p$ (ou bien $\C^p$).

Il y a une bijection naturelle entre l'ensemble des rev�tements ramifi�s de degr� $k$ de U, contenus dans $U \times B$, et l'ensemble des applications analytiques $f: U\to Sym^k(B)$.

\end{Prop}

\begin{proof}
cf \cite{\barlet} p 25
\end{proof}

\section{\'Ecailles}

\begin{Def}
Une $n$-�caille sur un espace analytique $X$ est la donn�e d'un triplet $E=(U,B,f)$ o� $U \subset \C^n,B \subset \C^p$ sont des polydisques relativement compacts et $f : V_E \to W$ est un hom�omorphisme d'un ouvert de $X$ dans un sous-espace analytique d'un voisinage $W$ de $\overline{U} \times \overline{B}$. L'ouvert $f^{-1}(U \times B)$ est appel� centre de l'�caille. \\
On dira qu'une �caille $(U,B,f)$ est adapt�e au $n$-cycle $Z$ si 
	\[f^{-1}(\overline{U} \times \partial B) \cap |Z|=\emptyset
\]
.
\end{Def}
Si $E=(U,B,f)$ est une �caille de $X$ et $Z=\sum n_i Z_i $ un cycle de $X$, on note par $f_*(Z \cap V_E)$ le cycle $\sum n_i f(V_E \cap Z_i)$. Si, de plus, $E$ est une �caille adapt�e � $X$ alors on a un rev�tement ramifi� $Z_E:=f_*(Z \cap V_E) \cap U \times B \to U$ (\cite{\cac} p 151). On notera par $\deg_E(X)$ le degr� de ce rev�tement ramifi�. 

\begin{Lemme}\label{ecailleadaptee}
Soit $X$ un espace analytique. Pour tout $n$-cycle complexe $Z$, tout point $z \in X$ et tout voisinage $V$ de $z$, il existe une $n$-�caille adapt�e � $Z$ dont le centre contient $z$ et dont le domaine est contenu dans $V$. Si $z \notin |Z|$, on peut choisir $E$ de sorte que $\deg_E(X)=0$.

\end{Lemme}
\begin{proof}
Cela vient du th�or�me de param�trisation locale cf. \cite{\cac} thm 3.5.4 page 154 et du caract�re local de la d�finition d'�caille adapt�e. 
\end{proof}
\begin{Rq}\label{revetramif}
Si l'�caille $E=(U,B,f)$ est adapt�e au cycle $Z$, alors $j_*(Z \cap V_E)$ d�finit un rev�tement ramifi� sur $U_1$, un polydisque ouvert suffisamment petit contenant $\overline{U}$ (tel que $U_1 \times \overline{B} \subset W$). En notant $k=\deg_E(Z)$, on peut donc associer � $Z$ un �l�ment $f_E \in \Hcal(\overline{U},Sym^k(B))$ tel que le cycle sous-jacent � son graphe co�ncide avec $Z_E$ sur $U \times B$ en restreignant l'application obtenue par \ref{bij_rr_appli} � $\overline{U}$ qui ne d�pend pas du voisinage de $\overline{U}$ choisie.
\end{Rq}
%cf \cite{\cac} Chap IV Rem iii) apr�s 2.1.3 
\section{Familles analytiques et espaces des cycles}
 \begin{Def}
Soient $X$ un espace analytique et $S$ un espace analytique r�duit. Une famille $(Z_s)_{s \in S}$ de $n$-cycles sera dite analytique\footnote{ dans \cite{\cac}, c'est analytique propre} au voisinage de $s_0 \in S$, s'il existe un ouvert $W$ relativement compact dans $X$ tel que, pour chaque $s \in S$ suffisamment proche de $s_0$, on ait $|Z_s| \subset W$ et si pour toute �caille $E=(U_E,B_E,f_E)$ adapt�e � $X_{s_0}$, il existe un voisinage ouvert $S_E$ de $s_0$ dans $S$ tel que : 
\begin{itemize} 
	\item $E$ soit adapt� � $X_s$ pour tout $s$ dans $S_E$.
	\item $\deg_E(X_s)=\deg_E(X_{s_0})$ pour tout $s$ dans $S_E$
	\item L'application $f_E : S_E \times U \to Sym^k(B_E)$ induite par les cycles $(j_* Z_s)_{s \in S_E}$ est holomorphe.
\end{itemize}
La famille $(X_s)_{s \in S}$ sera dite analytique si elle l'est en tout point.
\end{Def}
\begin{Lemme}
Si $f : S \to T$ est une fonction holomorphe et $(Z_t)_{t \in T}$ est une famille analytique alors $(Z_{f(s)})_{s \in S}$ est une famille analytique.
\end{Lemme}
On peut donc d�finir le foncteur contravariant $F_X^n$ des espaces complexes r�duits de dimension finie dans celui des ensembles qui associe � $S$ l'ensemble des familles analytiques de $n$-cycles param�tr�es par $S$. 

\begin{Thm}[Barlet]
$F_X^n$ est un foncteur repr�sentable par un espace analytique $\Cscr_n(X)$.
\end{Thm}
Pour exploiter ce r�sultat, on va utiliser le lemme de Yoneda : 

\begin{Thm}[Lemme de Yoneda]
Soit $F : \Ccal \to \Dcal$ un foncteur et $X \in Ob(\Ccal)$. Alors, on a une bijection naturelle, 
	\[ \Hom(\Hom(X,\--),F) \simeq F(X)
\]
entre les transformations naturelles entre $\Hom(X,\--)$ et $F$, et les �l�ments de $F(X)$
\end{Thm}
En particulier, si $F$ est repr�sentable par un objet $X$ i.e. s'il existe un isomorphisme de foncteurs $\Hom(X,\--) \to F$ alors, par ce lemme, on obtient un objet $s \in F(X)$. On dira que le couple $(X,s)$ est universel pour $F$. \\ % Le Stum
Dans notre cas, $F$ est un foncteur de cat�gorie oppos�e � celle des espaces analytiques r�duits de dimension finie � celle des ensembles. \\
Ainsi, il existe une famille analytique de cycles index�s par $\Cscr_n(M)$, $(C_s)_{s \in \Cscr_n(M)}$, telle que, pour tout espace complexes r�duits de dimension finie $S$ et toute famille analytique $(B_s)_{s \in S}$, il existe une unique application holomorphe $h : S \to \Cscr_n(M)$ telle que pour tout $s \in S$, $B_s=C_{h(s)}$. On appellera cette famille la famille universelle des $n$-cycles.

On en d�duit qu'il existe une bijection entre $\Cscr_n(X)$ et l'ensemble des $n$-cycles (en prenant $S$ un singleton). On peut ainsi munir l'ensemble des $n$-cycles d'une structure d'espace analytique r�duit de dimension finie.

\begin{Prop}
$\Cscr_n$ est un endofoncteur de la cat�gorie des espaces analytiques complexes r�duits de dimension finie
\end{Prop}
\begin{proof}
Soit $f : M \to N$ une application holomorphe. Alors, on lui associe l'application d'image directe $f_* : \Cscr_n(M) \to \Cscr_n(N)$ donn� en \ref{image_directe}. \\
Cette application est holomorphe par le th�or�me 3.5.4 p 459 de \cite{\cac}

\end{proof}
\section{Topologie de $\Cscr_n(X)$}
L'espace analytique $\Cscr_n(X)$ obtenu pr�c�dement est en particulier un espace topologique. Ces ouverts se d�crivent de la fa�on suivante :
\begin{Thm}
Un sous-ensemble $U$ de $\Cscr_n(X)$ est ouvert si, et seulement si, pour tout $Z_0 \in U$, il existe des �cailles $E_1,\ldots,E_m$ sur $M$ adapt�e � $Z_0$ et un voisinage ouvert $W$ de $Z_0$ tels que, si l'on pose $k_i:=\deg_{E_i}(Z_0)$ pour chaque $i \in \{1,\ldots,m\}$,on ait : 
	\[ \bigcap_{i=1}^m \{Z \in \Cscr_n(W) \mid E_i \text{ est adapt�e � } Z, \deg_{E_i}(Z)=k_i \} \subset U
\]
\end{Thm}

\begin{Rq} \label{convCloc}
Les ouverts $\Cscr^{loc}_n(X)$ sont donn�s par les unions quelconques d'intersections finies des 
	\[\Omega_k(E):= \{ Z \in \Cscr_n^{loc}(X) \mid E \text{ est adapt�e � } X, \deg_{E}(Z)=k \}
\]
 Par cons�quent, si $(Z_n)$ est une suite de $\Cscr_n(X)$ et $Z$ est un cycle compact de $M$ alors $Z_n$ converge vers $Z$ dans $\Cscr_n(X)$ si, et seulement si, $Z_n$ converge vers $Z$ dans $\Cscr_n^{loc}(X)$ et pour tout voisinage ouvert de $|Z|$ dans $X$, il existe un entier $N_W$ tel que pour tout $n \geq N_W$, $|Z_n| \subset W$
\end{Rq}

Les trois r�sultats qui viennent sont �nonc�s pour $\Cscr^{loc}_n$ pour la simplicit� mais gr�ce � la remarque pr�c�dente, on peut les traduire dans $\Cscr_n$.

\begin{Prop}\label{cont_cycle}
Soient $U \subset \C^n$, $B \subset \C^p$ deux polydisques ouverts relativement compacts et un entier$k>0$. Alors l'application 
	\[ \Hcal(\overline{U},Sym^k(B)) \to \Cscr^{loc}_n(U \times B), f \mapsto Z_f
\]
est continue. ($Z_f$ est le cycle que l'on a associ� � $f$ dans \ref{revetramif})

\end{Prop}
\begin{proof}
Voir \cite{\cac} p 386
\end{proof}
\begin{Thm} \label{produit}
Soient $X$ un espace analytique complexe r�duit et $(X_i)_{i \in I}$ un recouvrement ouvert de $X$. Alors, l'application induite par les restrictions 
	\[ res : \Cscr_n^{loc}(X) \to \prod_{i\in I} \Cscr_n^{loc}(X_i)
	\]
est un hom�omorphisme sur son image (qui est ferm�e).	 


\end{Thm}
\begin{proof}
Voir \cite{\cac} p 396
\end{proof}
\begin{Prop}\label{localisation}
Soit $(Z_m)_{ m \in \N}$ une suite de $\Cscr^{loc}_n(X)$ telle que pour chaque $x \in M$, il existe un voisinage ouvert $X_x$ tel que $(Z_m \cap X_x)_{m \in \N}$ soit une sous-suite convergente dans $\Cscr_n^{loc}(X_x)$. Alors, la suite converge dans $\Cscr_n^{loc}(X)$

\end{Prop}
\begin{proof}
Voir \cite{\cac} p 409


\end{proof}


\ifwhole
 \end{document}
\fi

\section{}
On se fixe une m�trique hermitienne $H$ et une 2-forme $\omega$ associ�e. Ce choix ne change pas le r�sultat suivant.
\begin{Thm}[de compacit�] \label{compact}

Soit $S$ un sous-ensemble de $\Cscr(X)$. $S$ est relativement compact si, et seulement si, il existe un compact contenant le support de chaque cycles de $S$, et si le volume de ces cycles est uniform�ment born�.

\end{Thm}

\begin{proof}
$\Rightarrow$ : Supposons $S$ compact. Par cons�quent, $S$ intersecte un nombre fini de composantes de $\Cscr(X)$.
On peut donc supposer que $S$ est inclus dans un seul d'entre eux, disons inclus dans $\Cscr_k(X)$. \\
Soit $(Z_s)_{s \in S}$ la famille de cycles analytiques donn�e par la restriction de la famille universelle. 

Soit $Y:= \{ (s,y) \in S \times X \mid y \in |Z_s| \}$ le ``cycle universel". La projection de $Y$ sur $S$ est propre (cf. \cite{\barlet} thm 1).
Ainsi, si on note $pr_1$ (resp. $pr_2$) la projection de $Y$ sur $S$ (resp. sur $X$), on obtient que $pr_1^{-1}(S)$ est compact et donc $pr_2(pr_1^{-1}(S))$ l'est aussi. Mais on peut voir que $pr_2(pr_1^{-1}(S))=\bigcup_{s \in S} |Z_s|$.
Ensuite, l'application $s \in S \mapsto \frac{1}{k!}\int_{Z_s} \omega^k$ est une application continue comme compos�e de l'application continue $X \mapsto \frac{1}{k!}\int_X \omega^k$ (cf \ref{int_cont} ) et de l'application continue $s \mapsto Z_s$. Comme $S$ est compact alors cette application est born�e, ce qui finit la preuve de l'implication.

$\Leftarrow$ : Soit $(Z_i)_{i \in S}$ une suite de cycles de $S$. Les $Z_i$ ne peuvent prendre qu'un nombre fini de dimensions ( born� par la dimension de $X$), on peut donc extraire une sous-suite de telle sorte qu'ils soient tous de la m�me dimension (que l'on notera $k$). \\
Soit $K$ un compact contenant tous les supports.
On va extraire une sous-suite de $Z_i$ de sorte que les supports $|Z_i|$ convergent, pour la topologie de Hausdorff vers un compact que l'on notera $|Z_0|$(c.f. \ref{Hausdorff} et plus pr�cis�ment la proposition \ref{compHaud} pour la compacit� de l'ensemble des compacts de $K$ muni de cette topologie).
On peut ensuite remarquer, en notant $Z_i=\sum m_{ij} Z_{ij}$ (avec $m_{ij} \neq 0$), que l'on a  
	\[ Vol(|Z_i|)=\sum Vol(Z_{ij}) \leq \sum m_{ij} Vol(Z_{ij})=Vol(Z_i) 
\]
C'est-�-dire, le volume des supports est uniform�ment born�e. On en d�duit, par le th�or�me \ref{Bishop} de Bishop, que $|Z_0|$ est un ensemble analytique.\\
Montrons maintenant qu'une sous-suite de $Z_i$ converge vers un cycle $Z_0$ pour la topologie de l'espace de Barlet :  \\
Soit ${U_i}$ un recouvrement de $|Z_0|$ par des ouverts de $X$ et pour chacun de ses ouverts une �caille adapt�e $E_i=(\Delta^k,\Delta^{n-k},j : U_i \to W \supset\overline{\Delta^k} \times \overline{\Delta^{n-k}})$ � $X$ (elles existent par le lemme \ref{ecailleadaptee}). Par d�finition d'�caille adapt�e, on a alors, pour tout $j$, 
	\[d(f_j(|Z_0| \cap U_j),\overline{\Delta^k} \times \partial \Delta^{n-k})>0
\]
Comme $\lim |Z_i|=|Z_0|$ alors, pour $i$ suffisament grand et pour tout $j$,
\[d(f_j(|Z_i| \cap U_j),\overline{\Delta^k} \times \partial \Delta^{n-k})>0
\]
Quitte � restreindre $U_j$, on a alors :
	\[|Z_i| \cap U_j \cap f_j^{-1}(\overline{\Delta^k} \times \partial \Delta^{n-k})=\emptyset
\]
On en d�duit que ${f_j}_*(|Z_i| \cap U_j)$ est un rev�tement ramifi� sur $\Delta^k$. De m�me, en notant $Z_i=\sum m_{ik} Z_{ik}$, on obtient que ${f_j}_*(Z_{ik} \cap U_j)$ est un rev�tement ramifi�. 
On peut donc voir ${f_j}_*(Z_{i} \cap U_j)$ comme un rev�tement ramifi� � $n_{ij}$ feuilles \footnote{Si on a un cycle $Z=\sum_i n_i Z_i$ et des rev�tements ramifi�s $Z_i \to \Delta^k$ � $m_i$ feuilles, on peut voir $Z$ comme un rev�tement ramifi� � $\sum n_i m_i$ feuilles}\\
Comme le volume des $Z_i$ est uniform�ment born� alors les $n_{ij}$ le sont aussi (car il y a un nombre fini de $j$ par compacit� de $|Z_0|$
Les $n_{ij}$ �tant des entiers, on peut extraire une sous-suite de la suite $Z_i$ de sorte que pour tout $j$, la suite $(n_{ij})_i$ soit constante. \\
Par la remarque \ref{revetramif}, on peut identifier $({f_j}_*(Z_{i} \cap U_j))_j$ avec un point $f_i$ de $B:=\prod_j \Hcal(\overline{\Delta^k},Sym^{n_j}(\Delta^{n-k})$.
 On peut plonger $Sym^{n_j}(\Delta^{n-k}$ dans un certain $\C^N$ (voir \ref{plongement}). Ensuite, on sait que, pour tout ouvert $U' \subset U$ relativement compact, il existe une sous-suite de $(f_i)$ qui converge uniform�ment sur $\overline{U'}$, par le th�or�me de Vitali. Et ainsi, les cycles associ�s de $\Cscr^{loc}_n(U' \times B)$ convergent (cf \ref{cont_cycle}). Par le th�or�me de localisation \ref{localisation}, on en d�duit la convergence dans $\Ccal^{loc}_n(U \times B)$
%Par le lemme 2.7.22, les points associ�s au cycle $Z_i$ converge dans $B$ et les cycles associ�s dans $\Cscr_n^{loc}(\Delta^k \times \Delta^{n-k})$\footnote{ A d�tailler} \\
On en d�duit que les $Z_i \cap U_j$ convergent dans $\Cscr_n^{loc}(U_j)$ (car les applications induites sont continues). Par le th�or�me \ref{produit}, la suite $Z_i$ converge dans $\Cscr_n^{loc}(\bigcup U_j) \hookrightarrow \Cscr_n^{loc}(X)$ vers $Z_0$ (car $|Z_0| \subset \bigcup U_j$).
Comme le support des $Z_i$ est inclus dans $K$ alors la suite converge dans $\Cscr_n(X)$ (par la remarque \ref{convCloc}).
\end{proof}

\begin{Cor}\label{Kah_compact}
Si $X$ est une vari�t� compacte de K�hler alors $S$ est relativement compact si, et seulement si, les $Z_s$ sont dans un nombre fini de classes de cohomologie.
\end{Cor}

\begin{proof}
$\Leftarrow$ : Supposons que les $Z_s$ soit dans un nombre fini de classes de cohomologie $\alpha_1,\ldots,\alpha_p$. Alors, pour tout $s \in S$, 
	\[ Vol(Z_s) \leq \max_i Vol(\alpha_i) < \infty
\]
(On pourra trouver des d�tails sur l'int�gration des classes de cohomologie dans \ref{int_cohom}
$\Rightarrow$ : Supposons $S$ est relativement compact et inclus dans un unique $\Cscr_n(M)$. \\
Alors, $S$ rencontre un nombre fini de composantes connexes de $\Cscr_n(M)$ et comme l'application $X \in \Cscr_n(M) \mapsto \int_X \omega^n \in \R$ est localement constante et donc $Z_s$ n'a qu'un nombre fini de classes de cohomologie.
\end{proof}
En particulier, les composantes irr�ductibles de $\Cscr(X)$ sont compactes pour $X$ une vari�t� de K�hler compacte.

\begin{Thm} \label{lemmefibre}
Soit $f : M \to N$ une application holomorphe propre et surjective entre deux espaces complexes irr�ductibles. Soit $n:=\dim(M)-\dim(N)$. Alors il existe une unique application m�romorphe, que l'on appelera l'application fibre de $f$, $\varphi : N \dashrightarrow \Cscr_n(M)$  qui sur un ouvert de Zariski dense de $N$ associe � $y$ le $n$-cycle r�duit $f^{-1}(y)$ de $M$.
\end{Thm}
\begin{proof}
\cite{\cac} p 486
\end{proof}
\begin{Cor} \label{merom}
Soient $S$ un espace normal, $Z$ un cycle de $S\times X$  tel que $\pi_1 : |Z| \to S$ soit propre et surjective. \\
Soit $S_0=\{ s \in S \mid \dim \pi^{-1}(s)=\dim|Z|-\dim S\}$. \\
L'application naturelle $S_0 \to \Cscr(X)$ (cf \cite{\barlet} Thm 1) s'�tend m�romorphiquement sur $S$.

\end{Cor}

\begin{proof}
On d�compose $|Z|$ en composantes irr�ductibles : $|Z|=\bigcup Z_i$. Les images r�ciproques sont elles-m�mes irr�ductibles. On a donc des applications $\pi_{1i} : \pi_1^{-1}(Z_i) \to Z_i$ surjectives et propres. 
On obtient par le th�or�me \ref{lemmefibre} que pour tout $i$, il existe une application m�romorphe $\varphi_i : f^{-1}(Z_i) \dashrightarrow \Cscr_{n_i}(Z_i)$. L'application d'image directe induite par l'inclusion $Z_i \hookrightarrow X$ (par \ref{image_directe}) permet de recoller toutes ces applications en une application m�romorphe $S \dashrightarrow \Cscr(X)$.
\end{proof}

\begin{Cor}
Soit $X$ un espace analytique compact normal. \\
Soit $\Rcal$ une relation d'�quivalence sur $X$ d�finissant un ensemble analytique $R$. Il existe un espace analytique $Q$ et une application m�romorphe dominante $\pi : X \dashrightarrow Q$ tel que, pour tout $x$ dans un ouvert de Zariski, on ait :
	\[ \overline{ \pi^{-1}(\pi(x))}=\{ y \in X \mid y \Rcal x\}
	\]

\end{Cor}
\begin{proof}
Comme l'ensemble analytique $R$ peut �tre vu comme un cycle (gr�ce � sa d�composition en irr�ductibles)et que l'on a une projection $p : |R|=R \to X$ qui est surjective et propre. On obtient, gr�ce au corollaire \ref{merom}, une application $X \dashrightarrow \Cscr(X)$ qui sur un ouvert de Zariski de $X$ envoie $x$ sur $p^{-1}(x)$. On d�finit $Q$ comme l'adh�rence de Zariski de l'image de cette application. On en d�duit ainsi le r�sultat voulu.

\end{proof}



\section{ Applications aux automorphismes et la th�orie de la d�formation}
\begin{Thm}
Soit $X$ un espace analytique complexe compact et $f$ un automorphisme de $X$. Soit $\Cscr_\Gamma$ la composante irr�ductible de l'espace des cycles contenant le graphe de $f$. Alors, l'ensemble des points de $\Cscr_\Gamma$ correspondant aux automorphismes de $X$ est un ouvert de Zariski.
\end{Thm}
\begin{proof} 
 On supposera $X$ lisse. \\
Soit $Z \subset X \times X \times \Cscr_\Gamma$ le cycle universel.  La projection naturelle $\pi_3 : Z \to \Cscr_\Gamma$ est propre (cf \cite{\barlet} th�or�me 1).\\
Quitte � remplacer $\Cscr_\Gamma$ par un ouvert de Zariski, on peut supposer $\Cscr_\Gamma$ et $Z$ lisse. En effet, les lieux singuliers de $\Cscr_\Gamma$ et $Z$ sont des espaces analytiques. L'image par $\pi_3$ de ce dernier est aussi un espace analytique et donc son compl�mentaire est un ouvert de Zariski. L'image r�ciproque de cet ouvert est donc lisse.
Par la proposition 1.21 p 108 de \cite{\scv}, il existe un ouvert de Zariski $U$ tel que $\pi_{3\mid U}$ soit une submersion. Par les m�mes consid�rations que pour la lissit� de $Z$, $\pi_3$ est lisse sur un ouvert de Zariski de $\Cscr_\Gamma$. Ainsi, toutes les fibres sur cet ouvert sont lisses.
De la m�me fa�on, pour les projections $\pi_i : Z \to X$, on obtient que, sur un ouvert de Zariski $S_i$ de $Z$, ce sont des submersions. En tout point $z$ de cet ouvert, on a un isomorphisme entre le quotient  $T_z Z$ par $T_{z} \pi_i^{-1}(\pi_i(x))$ et $T_{\pi_i(x)} X$ (par compl�tion de la base de $T_{z} \pi_i^{-1}(\pi_i(x))$)  et donc un isomorphisme entre $T_z Z_C$ et $T_{\pi_i(z)} X$ pour tout $z=(x,y,C) \in Z$  tel que $(x,y) \in C$ car l'application $\pi_i \times \pi_3 : Z \to X\times \Cscr$ est � fibres finies (quitte � restreindre) car $X$ est compact et que $\dim Z=\dim X + \dim \Cscr$. \\
Pour finir, montrons que les points de $\Cscr_\Gamma \setminus \pi_3(S_1^c \cup S_2^c)$ correspondent aux automorphismes de $X$. \\
Soit $C$ un cycle de $\Cscr_\Gamma$. \\
Les applications $\pi_i : Z_C \to X$ est un rev�tement propre par \ref{}. \\
Comme $\Cscr_\Gamma$ est connexe et localement connexe par arcs alors il est connexe par arcs. Il existe donc un chemin $\gamma : [0,1] \to \Cscr_\Gamma$ entre $C$ et $\Gamma$. Par continuit� de la famille $(\gamma(t))_{t \in [0,1]}$ le degr� du rev�tement reste constant ( au voisinage de chacun des points de $X$) et donc vaut toujours 1. $\pi_i$ est donc un hom�omorphisme et donc par l'isomorphisme pr�c�dent et le th�or�me d'inversion locale, c'est un biholomorphisme.

\end{proof}

\appendix 


\newif\ifwhole

%\wholetrue
% Ajouter \wholetrue si on compile seulement ce fichier

\ifwhole
 \documentclass[a4page,10pt]{article}
     \usepackage[Latin1]{inputenc}
\usepackage[francais]{babel}
\usepackage{amsmath,amssymb,amsthm}
\usepackage{textcomp}
\usepackage{mathrsfs}
\usepackage{algcompatible,algorithm  }
\usepackage[all]{xy}
\usepackage{hyperref}
\usepackage{fancyhdr}
\usepackage{supertabular}
\usepackage{makeidx}
\usepackage{setspace}
\usepackage{makeidx}
\usepackage[Bjornstrup]{fncychap}
\usepackage[nottoc,notlot,notlof]{tocbibind}
\makeindex
\pagestyle{fancy}
\fancyhead[L]{\leftmark}
\fancyhead[R]{}
\onehalfspacing

\toks0=\expandafter{\xy}
\edef\xy{\noexpand\shorthandoff{!?;:}\the\toks0 }
\makeatletter

\renewcommand*{\ALG@name}{Algorithme}

\makeatother

\renewcommand{\algorithmicrequire}{\textbf{\textsc {Entr\'ees  :  } } }
\renewcommand{\algorithmicensure}{\textbf{\textsc { Sortie  :  } } }
\renewcommand{\algorithmicwhile}{\textbf{Tant que}}
\renewcommand{\algorithmicdo}{\textbf{faire }}
\renewcommand{\algorithmicif}{\textbf{Si}}
\renewcommand{\algorithmicelse}{\textbf{Sinon}}
\renewcommand{\algorithmicthen}{\textbf{alors }}
\renewcommand{\algorithmicend}{\textbf{fin}}
\renewcommand{\algorithmicfor}{\textbf{Pour}}
\renewcommand{\algorithmicuntil}{\textbf{Jusqu'\`a}}
\renewcommand{\algorithmicrepeat}{\textbf{Répéter}}

\newcommand{\A}{\mathbb{A}}
\newcommand{\B}{\mathbb{B}}
\newcommand{\C}{\mathbb{C}}
\newcommand{\D}{\mathbb{D}}
\newcommand{\E}{\mathbb{E}}
\newcommand{\F}{\mathbb{F}}
\newcommand{\G}{\mathbb{G}}
\renewcommand{\H}{\mathbb{H}}
\newcommand{\I}{\mathbb{I}}
\newcommand{\J}{\mathbb{J}}
\newcommand{\K}{\mathbb{K}}
\renewcommand{\L}{\mathbb{L}}
\newcommand{\M}{\mathbb{M}}
\newcommand{\N}{\mathbb{N}}
\renewcommand{\O}{\mathbb{O}}
\renewcommand{\P}{\mathbb{P}}
\newcommand{\Q}{\mathbb{Q}}
\newcommand{\R}{\mathbb{R}}
\renewcommand{\S}{\mathbb{S}}
\newcommand{\T}{\mathbb{T}}
\newcommand{\U}{\mathbb{U}}
\newcommand{\V}{\mathbb{V}}
\newcommand{\W}{\mathbb{W}}
\newcommand{\X}{\mathbb{X}}
\newcommand{\Y}{\mathbb{Y}}
\newcommand{\Z}{\mathbb{Z}}
\newcommand{\Acal}{\mathcal{A}}
\newcommand{\Bcal}{\mathcal{B}}
\newcommand{\Ccal}{\mathcal{C}}
\newcommand{\Dcal}{\mathcal{D}}
\newcommand{\Ecal}{\mathcal{E}}
\newcommand{\Fcal}{\mathcal{F}}
\newcommand{\Gcal}{\mathcal{G}}
\newcommand{\Hcal}{\mathcal{H}}
\newcommand{\Ical}{\mathcal{I}}
\newcommand{\Jcal}{\mathcal{J}}
\newcommand{\Kcal}{\mathcal{K}}
\newcommand{\Lcal}{\mathcal{L}}
\newcommand{\Mcal}{\mathcal{M}}
\newcommand{\Ncal}{\mathcal{N}}
\newcommand{\Ocal}{\mathcal{O}}
\newcommand{\Pcal}{\mathcal{P}}
\newcommand{\Qcal}{\mathcal{Q}}
\newcommand{\Rcal}{\mathcal{R}}
\newcommand{\Scal}{\mathcal{S}}
\newcommand{\Tcal}{\mathcal{T}}
\newcommand{\Ucal}{\mathcal{U}}
\newcommand{\Vcal}{\mathcal{V}}
\newcommand{\Wcal}{\mathcal{W}}
\newcommand{\Xcal}{\mathcal{X}}
\newcommand{\Ycal}{\mathcal{Y}}
\newcommand{\Zcal}{\mathcal{Z}}
\newcommand{\Ascr}{\mathscr{A}}
\newcommand{\Bscr}{\mathscr{B}}
\newcommand{\Cscr}{\mathscr{C}}
\newcommand{\Dscr}{\mathscr{D}}
\newcommand{\Escr}{\mathscr{E}}
\newcommand{\Fscr}{\mathscr{F}}
\newcommand{\Gscr}{\mathscr{G}}
\newcommand{\Hscr}{\mathscr{H}}
\newcommand{\Iscr}{\mathscr{I}}
\newcommand{\Jscr}{\mathscr{J}}
\newcommand{\Kscr}{\mathscr{K}}
\newcommand{\Lscr}{\mathscr{L}}
\newcommand{\Mscr}{\mathscr{M}}
\newcommand{\Nscr}{\mathscr{N}}
\newcommand{\Oscr}{\mathscr{O}}
\newcommand{\Pscr}{\mathscr{P}}
\newcommand{\Qscr}{\mathscr{Q}}
\newcommand{\Rscr}{\mathscr{R}}
\newcommand{\Sscr}{\mathscr{S}}
\newcommand{\Tscr}{\mathscr{T}}
\newcommand{\Uscr}{\mathscr{U}}
\newcommand{\Vscr}{\mathscr{V}}
\newcommand{\Wscr}{\mathscr{W}}
\newcommand{\Xscr}{\mathscr{X}}
\newcommand{\Yscr}{\mathscr{Y}}
\newcommand{\Zscr}{\mathscr{Z}}
\newcommand{\Afrak}{\mathfrak{A}}
\newcommand{\Bfrak}{\mathfrak{B}}
\newcommand{\Cfrak}{\mathfrak{C}}
\newcommand{\Dfrak}{\mathfrak{D}}
\newcommand{\Efrak}{\mathfrak{E}}
\newcommand{\Ffrak}{\mathfrak{F}}
\newcommand{\Gfrak}{\mathfrak{G}}
\newcommand{\Hfrak}{\mathfrak{H}}
\newcommand{\Ifrak}{\mathfrak{I}}
\newcommand{\Jfrak}{\mathfrak{J}}
\newcommand{\Kfrak}{\mathfrak{K}}
\newcommand{\Lfrak}{\mathfrak{L}}
\newcommand{\Mfrak}{\mathfrak{M}}
\newcommand{\Nfrak}{\mathfrak{N}}
\newcommand{\Ofrak}{\mathfrak{O}}
\newcommand{\Pfrak}{\mathfrak{P}}
\newcommand{\Qfrak}{\mathfrak{Q}}
\newcommand{\Rfrak}{\mathfrak{R}}
\newcommand{\Sfrak}{\mathfrak{S}}
\newcommand{\Tfrak}{\mathfrak{T}}
\newcommand{\Ufrak}{\mathfrak{U}}
\newcommand{\Vfrak}{\mathfrak{V}}
\newcommand{\Wfrak}{\mathfrak{W}}
\newcommand{\Xfrak}{\mathfrak{X}}
\newcommand{\Yfrak}{\mathfrak{Y}}
\newcommand{\Zfrak}{\mathfrak{Z}}
\newcommand{\afrak}{\mathfrak{a}}
\newcommand{\bfrak}{\mathfrak{b}}
\newcommand{\cfrak}{\mathfrak{c}}
\newcommand{\dfrak}{\mathfrak{d}}
\newcommand{\efrak}{\mathfrak{e}}
\newcommand{\ffrak}{\mathfrak{f}}
\newcommand{\gfrak}{\mathfrak{g}}
\newcommand{\hfrak}{\mathfrak{h}}
\newcommand{\ifrak}{\mathfrak{i}}
\newcommand{\jfrak}{\mathfrak{j}}
\newcommand{\kfrak}{\mathfrak{k}}
\newcommand{\lfrak}{\mathfrak{l}}
\newcommand{\mfrak}{\mathfrak{m}}
\newcommand{\nfrak}{\mathfrak{n}}
\newcommand{\ofrak}{\mathfrak{o}}
\newcommand{\pfrak}{\mathfrak{p}}
\newcommand{\qfrak}{\mathfrak{q}}
\newcommand{\rfrak}{\mathfrak{r}}
\newcommand{\sfrak}{\mathfrak{s}}
\newcommand{\tfrak}{\mathfrak{t}}
\newcommand{\ufrak}{\mathfrak{u}}
\newcommand{\vfrak}{\mathfrak{v}}
\newcommand{\wfrak}{\mathfrak{w}}
\newcommand{\xfrak}{\mathfrak{x}}
\newcommand{\yfrak}{\mathfrak{y}}
\newcommand{\zfrak}{\mathfrak{z}}
\newcommand{\Abar}{\overline{A}}
\newcommand{\Bbar}{\overline{B}}
\newcommand{\Cbar}{\overline{C}}
\newcommand{\Dbar}{\overline{D}}
\newcommand{\Ebar}{\overline{E}}
\newcommand{\Fbar}{\overline{F}}
\newcommand{\Gbar}{\overline{G}}
\newcommand{\Hbar}{\overline{H}}
\newcommand{\Ibar}{\overline{I}}
\newcommand{\Jbar}{\overline{J}}
\newcommand{\Kbar}{\overline{K}}
\newcommand{\Lbar}{\overline{L}}
\newcommand{\Mbar}{\overline{M}}
\newcommand{\Nbar}{\overline{N}}
\newcommand{\Obar}{\overline{O}}
\newcommand{\Pbar}{\overline{P}}
\newcommand{\Qbar}{\overline{Q}}
\newcommand{\Rbar}{\overline{R}}
\newcommand{\Sbar}{\overline{S}}
\newcommand{\Tbar}{\overline{T}}
\newcommand{\Ubar}{\overline{U}}
\newcommand{\Vbar}{\overline{V}}
\newcommand{\Wbar}{\overline{W}}
\newcommand{\Xbar}{\overline{X}}
\newcommand{\Ybar}{\overline{Y}}
\newcommand{\Zbar}{\overline{Z}}
\newcommand{\abar}{\overline{a}}
\newcommand{\bbar}{\overline{b}}
\newcommand{\cbar}{\overline{c}}
\newcommand{\dbar}{\overline{d}}
\newcommand{\ebar}{\overline{e}}
\newcommand{\fbar}{\overline{f}}
\newcommand{\gbar}{\overline{g}}
\renewcommand{\hbar}{\overline{h}}
\newcommand{\ibar}{\overline{i}}
\newcommand{\jbar}{\overline{j}}
\newcommand{\kbar}{\overline{k}}
\newcommand{\lbar}{\overline{l}}
\newcommand{\mbar}{\overline{m}}
\newcommand{\nbar}{\overline{n}}
\newcommand{\obar}{\overline{o}}
\newcommand{\pbar}{\overline{p}}
\newcommand{\qbar}{\overline{q}}
\newcommand{\rbar}{\overline{r}}
\newcommand{\sbar}{\overline{s}}
\newcommand{\tbar}{\overline{t}}
\newcommand{\ubar}{\overline{u}}
\newcommand{\vbar}{\overline{v}}
\newcommand{\wbar}{\overline{w}}
\newcommand{\xbar}{\overline{x}}
\newcommand{\ybar}{\overline{y}}
\newcommand{\zbar}{\overline{z}}
\newcommand\bigzero{\makebox(0,0){\text{\huge0}}}
\newcommand{\limp}{\lim\limits_{\leftarrow}}
\newcommand{\limi}{\lim\limits_{\rightarrow}}


\DeclareMathOperator{\End}{\mathrm{End}}
\DeclareMathOperator{\Hom}{\mathrm{Hom}}
\DeclareMathOperator{\Vect}{\mathrm{Vect}}
\DeclareMathOperator{\Spec}{\mathrm{Spec}}
\DeclareMathOperator{\multideg}{\mathrm{multideg}}
\DeclareMathOperator{\LM}{\mathrm{LM}}
\DeclareMathOperator{\LT}{\mathrm{LT}}
\DeclareMathOperator{\LC}{\mathrm{LC}}
\DeclareMathOperator{\PPCM}{\mathrm{PPCM}}
\DeclareMathOperator{\PGCD}{\mathrm{PGCD}}
\DeclareMathOperator{\Syl}{\mathrm{Syl}}
\DeclareMathOperator{\Res}{\mathrm{Res}}
\DeclareMathOperator{\Com}{\mathrm{Com}}
\DeclareMathOperator{\GL}{\mathrm{GL}}
\DeclareMathOperator{\SL}{\mathrm{SL}}
\DeclareMathOperator{\SU}{\mathrm{SU}}
\DeclareMathOperator{\PGL}{\mathrm{PGL}}
\DeclareMathOperator{\PSL}{\mathrm{PSL}}
\DeclareMathOperator{\PSU}{\mathrm{PSU}}
\DeclareMathOperator{\SO}{\mathrm{SO}}
\DeclareMathOperator{\Sp}{\mathrm{Sp}}
\DeclareMathOperator{\Spin}{\mathrm{Spin}}
\DeclareMathOperator{\Ker}{\mathrm{Ker}}
%\DeclareMathOperator{\Im}{\mathrm{Im}}

\DeclareMathOperator{\Ens}{\mathbf{Ens}}
\DeclareMathOperator{\Top}{\mathbf{Top}}
\DeclareMathOperator{\Ann}{\mathbf{Ann}}
\DeclareMathOperator{\Gr}{\mathbf{Gr}}
\DeclareMathOperator{\Ab}{\mathbf{Ab}}
%\DeclareMathOperator{\Vect}{\mathbf{Vect}}
\DeclareMathOperator{\Mod}{\mathbf{Mod}}
\headheight=0mm
\topmargin=-20mm
\oddsidemargin=-1cm
\evensidemargin=-1cm
\textwidth=18cm
\textheight=25cm
\parindent=0mm
\newif\ifproof
\newcommand{\demo}[1]{\ifproof #1 \else \fi}
 %Instruction d'utilisation : 
%les preuves du texte sont, en principe, entre des balises \demo, en sus des \begin{proof} pour l'instant.
%Laisser le texte tel quel, fait qu'elles ne sont pas affich�es.
%Mettre \prooftrue fait que toutes les preuves jusqu'� un \prooffalse ou la fin du document. 


 \begin{document}
\newtheorem{Thm}{Th�or�me}[chapter]
\newtheorem{Prop}[Thm]{Proposition}
\newtheorem{Propte}[Thm]{Propri�t�}
\newtheorem{Lemme}[Thm]{Lemme}
\newtheorem{Cor}[Thm]{Corollaire}


\theoremstyle{definition}

\newtheorem{Ex}[Thm]{Exemple}
\newtheorem{Def}[Thm]{D�finition}
\newtheorem{Defpropte}[Thm]{D�finition et propri�t�}
\newtheorem{Defprop}[Thm]{D�finition et proposition}
\newtheorem{Defthm}[Thm]{Th�or�me et d�finition}
\newtheorem{Not}[Thm]{Notation}
\newtheorem{Conv}[Thm]{Convention}
\newtheorem{Cons}[Thm]{Construction}

\theoremstyle{remark}
\newtheorem{Rq}[Thm]{Remarque}
\newtheorem{Slog}[Thm]{Slogan}
\newtheorem{Exo}[Thm]{Exercice}

\fi

\section{Topologie g�n�rale}
\subsection{Paracompacit�}
\begin{Def}
Un espace topologique est paracompact s'il est s�par� et si, pour tout recouvrement ouvert de $X$, il existe un recouvrement ouvert plus fin de $X$ qui est localement fini.
\end{Def}
\begin{Def}
Soit $X$ un espace localement compact. $X$ est dit d�nombrable � l'infini si le point � l'infini de sa compactification d'Alexandrov admet une base d�nombrable de voisinage.
\end{Def}
\begin{Prop}
Soit $X$ un espace localement compact.
$X$ est d�nombrable � l'infini si, et seulement, $X$ est l'union d�nombrable de compacts. 
\end{Prop}
\begin{proof}
$ \Rightarrow $ : Soit $(V_i)_{i \in \N}$ la base de voisinage du point � l'infini $\infty$. \\
Montrons que 
	\[X=\bigcup_{n \in \N} V_i^c
\]
Soit $x \in X$. Comme $X \cup \{\infty\}$ est s�par�, il existe des voisinages $V$ de $x$ et $W$ de $\infty$ qui ne rencontrent pas. Ainsi, $x \in W^c$ et par d�finition de  bases de voisinages, il existe $i \in \N$ tel que $V_i \subset W$ et donc $W^c \subset V_i^c$, ce qui montre que $X \subset \bigcup_{n \in \N} V_i^c$. L'autre inclusion vient du fait que $\infty \in V_i$ pour $i$. \\
$ \Leftarrow $ : Soit $(K_n)$ une suite de compacts de $X$ tel que : 
	\begin{equation} \label{denalinfini}
	X=\bigcup_{n \in \N} K_n
\end{equation}
Pour tout $n \in \N$, on notera par $B_n$ un voisinage compact de $K_n$ et $C_n=\bigcup_{k=0}^n B_k$. \\
La suite $(C_n)$ est une suite croissante pour l'inclusion et 
\[ X=\bigcup_{n \in \N} Int(C_n)
\]
Soit $V$ un voisinage quelconque de $\infty$ dans $X \cup \infty$. Il existe donc un voisinage $U$ de $\infty$ tel que $U \subset V$. Par d�finition de la topologie du compactifi�, il existe un compact $K$ tel que $U=K^c$. \\
Par l'�galit� \ref{denalinfini} et le fait que la suite $(C_n)$ est croissante, il existe un $N$ tel que $K \subset Int(C_n)$. On en d�duit que : 
	\[Int(C_n)^c \subset K^c=U \subset V
\]
\end{proof}

%\begin{Lemme}
%Soit $X$ un espace localement connexe. $X$ est la somme de ces composantes connexes.
%\end{Lemme}
%
%\begin{proof}
%\cite{btg} p 85
%\end{proof}

\begin{Lemme}
Soit $X$ un espace localement connexe. $X$ est d�nombrable � l'infini si, et seulement si, $X$ est paracompact.
\end{Lemme}

\begin{proof}
Cela d�coule de \cite{\btg} p 71 et \cite{\btg} p 85
\end{proof}


\begin{Thm}[Smirnov]
Un espace est m�trisable si, et seulement si, il est paracompact et localement m�trisable
\end{Thm}

\begin{Cor}
Un espace analytique est m�trisable.

\end{Cor}
\subsection{ Distance de Hausdorff}
\label{Hausdorff}
Soit $(M,d)$ un espace m�trique, $K(M)$ l'ensemble des compacts non vide de $M$. \\
On appelle distance de Hausdorff de cet espace m�trique l'application $d_H : K(M) \times K(M) \to \R_+$ donn�e par :
	\[ d_H(A,B)=\frac{1}{2}\left[\max\limits_{b \in B} d(A,b)+\max\limits_{a \in A} d(a,B)\right]
\]
\begin{Prop}
Soit $M$ un espace topologique m�trisable. Pour toute distance sur $M$ d�finissant la topologie de $M$, la distance de Hausdorff associ�e d�finit la m�me topologie sur $K(M)$. 
\end{Prop}
\begin{proof}
cf \cite{\cac} p 419
\end{proof}
On appelera cette topologie la topologie de Hausdorff sur $K(M)$
\begin{Prop} \label{compHaud}
Soit $M$ un espace topologique m�trisable compact. Alors $K(M)$ est un espace topologique compact.
\end{Prop}

\begin{Thm}[Bishop] \label{Bishop}
Soit $(A_m)_{m \in \N}$ une suite de sous-ensembles analytiques (ferm�s) de dimension $n$ d'un ouvert $W$ de $\C^{n+p}$, convergeant au sens de Hausdorff de $W$ pour la m�trique standard de $\C^{n+p}$ vers un ensemble ferm� $A$ de $W$. Si le volume des $A_m$ est fini et uniform�ment major�, $A$ est un sous-ensemble analytique de $W$ de dimension $n$.
\end{Thm}

\section{Analyse fonctionnelle}

\begin{Def}%Brezis
Soient $E,F$ deux espaces de Banach et $L : E \to F$ une application lin�aire continue. On dit que $L$ est compacte si l'image par $L$ de la boule unit� de $E$ est relativement compacte dans $F$.
\end{Def}
\begin{Not}
Soit $U \subset \C^n$ un ouvert relativement compact de $\C^n$ et $F$ un $\C$-espace vectoriel norm� de dimension finie. On note $\Hcal(\overline{U},F)$ l'espace de Banach des applications continues $\overline{U} \to F$, holomorphes sur $U$ muni de la norme $\| f\|=\sup\limits_{x \in \overline{U}} \|f(x)\|_F$ 
\end{Not}
\begin{Thm}[Vitali]
Soit $U \subset \C^n$ et $U' \subset U$ un ouvert relativement compact de $U$. Alors l'application de restriction 
	\[res : \Hcal(\overline{U},\C) \to \Hcal(\overline{U'},\C)
\]
est une application lin�aire continue compacte.
\end{Thm}
\begin{proof}
Cf p13 \cite{\cac}
\end{proof}
\section{ Volumes}
$M$ d�signera une vari�t� de classe $\Ccal^\infty$ de dimension (r�elle) $n$.
\begin{Def}
Une forme volume sur $M$ est une section  de $\bigwedge^n T^*M$ qui ne s'annule jamais.
\end{Def}
\begin{Prop}
$M$ admet une forme volume  si, et seulement si, $M$ est orientable
\end{Prop}
\begin{Prop}
Soit $(M,g)$ une vari�t� riemannienne orientable. $M$ admet une forme volume naturelle :  en coordonn�es locales, elle est donn�e par : 
	\[Vol_g=\sqrt{\det(g_{ij})} dx_1 \wedge \ldots \wedge dx_n
\]
o� $g_{ij}: x \mapsto g(\frac{\partial}{\partial x_i},\frac{\partial}{\partial x_j})(x)$. \\
Cette forme volume v�rifie la propri�t� suivante : \\
Pour toute point $x$ de $M$ et toute base orthonorm�e $(e_1,\ldots,e_n)$ de $T_xM$, on a l'�galit� : 
	\[Vol_g(x)(e_1 \wedge \ldots \wedge e_n)=1
\]
\end{Prop}
\begin{Prop}
Soit $(M,\omega)$ une vari�t� symplectique. Alors $\omega^n$ est une forme volume sur $M$ ($M$ est donc orient�e)
\end{Prop}
\begin{Thm}
Soit $(M,h)$ une vari�t� presque complexe muni d'une m�trique hermitienne. La forme $g=Re(h)$ est une m�trique riemannienne. De plus, en notant $\omega=-Im(h)$, on obtient l'�galit� : 
	\[ Vol_g=\frac{\omega^n}{n !}
\]
\end{Thm}
\begin{proof}
cf \cite{\voisin}
\end{proof}
\begin{Prop}
Soit $(X,\omega)$ une vari�t� de K�hler. Alors, pour tout $k=1..n$, $\omega^k$ est ferm� mais pas exacte.
\end{Prop}
\begin{proof}
Cela se montre par r�currence (la d�monstration du premier fait n'utilise pas celle du deuxi�me) : \\
Pour $k=1$, $\omega$ est ferm�e (par d�finition) mais n'est pas exacte car : \\
Si $f : M \to \C$ est une fonction telle que $\omega=df$ alors $\omega^n=d(f \omega^{n-1})$ est exacte donc $Vol(M)=0$, ce qui est absurde. \\
Supposons la propri�t� vraie pour $k \in \{1,\ldots,d-1\}$ alors : \\
$d\omega^{k+1}=d\omega \wedge \omega^k \pm \omega \wedge d\omega^k=0$ (par l'hypoth�se de r�currence : $\omega^k$ est ferm�e). \\
Si on suppose $\omega^{k+1}=d \gamma$ alors $d(\gamma \wedge \omega^{n-k-1})=\omega^n\pm \gamma d\omega^{n-k-1}=\omega^n$, ce qui est absurde.
\end{proof}

\begin{Thm}[Wirtinger]
Soit $(M,h)$ vari�t� hermitienne avec $\omega=-Im(h)$ la forme associ�e. Soit $S$ une sous-vari�t� de $M$ de dimension $k$. Alors, 
	\[ Vol(S)=\frac{1}{k !}\int_S \omega^k
\]
\end{Thm}

\begin{proof}
cf \cite{\pag} p31
\end{proof}

On peut �tendre ce r�sultat aux sous-espaces analytiques en d�finissant l'int�grale sur un sous-espace analytique comme �tant l'int�grale sur son ouvert lisse. 

\subsection{Volume d'un graphe}
Soient $(X,\omega_1)$ et $(Y,\omega_2)$ deux vari�t�s de K�hler. Alors en posant pour tout $(x,y) \in X \times Y$, $\omega_{(x,y)}=\omega_{1,x}+\omega_{2,y}$, on obtient que $(X \times Y,\omega)$ est une vari�t� de K�hler. 
\begin{Lemme} \label{Vol_graphe}
Soit $f : X \to Y$ une fonction holomorphe. Alors, 
	\[ Vol(\Gamma_f)=\frac{1}{n ! }\int_X (\omega_1+f^* \omega_2)
\]
\end{Lemme}

\begin{proof}
Par la formule de Wirtinger, en posant $n:=\dim(X)$, 
	\[Vol(\Gamma_f)=\frac{1}{n ! }\int_{\Gamma_f} \omega^n
\]
La projection $\pi$ de $\Gamma_f$ sur $X$ est un diff�omorphisme de r�ciproque $x \mapsto (x,f(x))$. On en d�duit que : 
\[Vol(\Gamma_f)=\frac{1}{n ! }\int_{X} \pi^*\omega^n
\]
Pour tout $x \in X$, $v,w \in  T_x X$, 
	\[\pi^*\omega_x(v,w)=\omega_{(x,f(x))}((v,d_x f(v)),(w,d_x f(w)))=\omega_{1,x}(v,w)+\omega_{2,f(x)}(d_x f(v),d_x f(w))
	\]
	Ainsi, $\pi^*\omega=\omega_1+f^* \omega_2$. Ce qui permet de conclure.
\end{proof}
\subsection{ Int�grations et classes de cohomologie}
\label{int_cohom}
\begin{Thm}[Lemme de Poincar�] \label{lem_poinc}
Soit $M$ une vari�t� diff�rentielle orient�e et soit $k$ un entier, alors on a un isomorphisme
	\[ H^k_{dR}(M) \simeq H^{n-k}_{dR}(M)^*
\]
donn� par $[\varphi] \mapsto ([\psi] \mapsto \int_M \varphi \wedge \psi)$.

\end{Thm}
\begin{proof}
\cite{\pag}
\end{proof}
Soit $M$ une vari�t� complexe compacte, $V \subset M$ un sous-ensemble analytique de dimension $k$. On peut d�finir une application lin�aire :
	\[ H^{2k}_{dR}(M) \to \R, [\varphi] \mapsto \int_V \varphi
	\]
Cette application est bien d�finie car les formes exactes sont envoy�es sur 0 par le th�or�me de Stokes : 	 
\begin{Thm}[Stokes pour les espaces analytiques]
Soit $M$ une vari�t� complexe, $V$ un sous-ensemble analytique de dimension $k$, et $\varphi$ une forme diff�rentielle de degr� $2k-1$ � support compact de $M$,
	\[\int_V d\varphi=0
\]

\end{Thm}
\begin{proof}
cf \cite{\pag} p33
\end{proof}

Par le lemme de Poincar� \ref{lem_poinc}, cette application d�termine un �l�ment de $H^{2n-2k}_{dR}(M)$ que l'on appelera classe (fondamentale) de $V$ et que l'on notera $[V]$. On d�finit la classe $[Z]$ d'un cycle analytique $Z=\sum_i n_i Z_i$ d'une vari�t� $M$ comme �tant $\sum n_i [Z_i]$

Dans le cas o� $(M,\omega)$ est une vari�t� de K�hler, on peut d�finir le volume d'une classe de cohomologie qui g�n�ralise celui des sous-espaces analytiques :
\begin{Def}
Soit $\alpha \in H^k_{dR}(M)$. On appelle volume de $\alpha$ le r�el $Vol(\alpha)=\frac{1}{n !}\int_M \alpha \wedge \omega^{n-k}$.
\end{Def}
Si $V$ est un sous-espace analytique de $M$ alors par d�finition de la classe de $V$, 
	\[Vol([V])=\frac{1}{n !}\int_V \omega^{n-k}=Vol(V)
\]
De plus, si $Z=\sum_i n_i Z_i$ est un cycle  de $M$, $Vol([Z])=\sum n_i Vol([Z_i])=\sum n_i Vol(Z_i)$ est appel� volume du cycle $Z$.

On va donner deux r�sultats sur l'int�gration sur les cycles que nous utilisons dans le corps du texte : 
\begin{Prop}\label{int_cont}
Soit $M$ une vari�t� de K�hler\footnote{ A voir s'il faut donner le r�sultat dans toute sa g�n�ralit�}. 
Alors, la fonction $X \in \Cscr_n(M) \mapsto \int_X \omega^n$ est continue.

\end{Prop}
\begin{proof}
cf \cite{\cac} p 389
\end{proof}

\begin{Thm} \label{int_const}
Soient $(M,\omega)$ une vari�t� de K�hler\footnote{idem}. Alors, la fonction $X \in \Cscr_n(M) \mapsto \int_X \omega^n$ est localement constante.
\end{Thm}
\begin{proof}
cf \cite{\cac} p 409
\end{proof}
\section{ Espace normal et faiblement normal}
\begin{Def}
Soient $B$ un anneau commutatif et $A$ est un sous-anneau de $B$. On dira que $b \in B$ est entier sur $A$ s'il existe un polyn�me unitaire $P \in A[X]$ tel que $P(b)=0$
\end{Def}

\begin{Def}
Un anneau commutatif $A$ est dit int�gralement clos s'il est int�gre et si tout �l�ment du corps de fraction $Frac(A)$ qui est entier sur $A$ est dans $A$.
\end{Def}

\begin{Def}
Un espace analytique complexe r�duit $X$ sera dit normal en $a \in X$ si l'anneau local $\Ocal_{X,a}$ est int�gralement clos. On dira que $X$ est normal s'il est normal en chacun de ses points.

\end{Def}
En particulier, un espace normal en $a$ est irr�ductible en $a$.

\subsection{ Fonctions m�romorphes}


Soit $X$ un espace analytique complexe r�duit. On d�finit les fonctions m�romorphes sur $X$ comme suit :  
\begin{Def}
Une fonction m�romorphe sur $X$ est la donn�e d'une classe d'�quivalence $(H,f)$ o� $H$ est un sous-ensemble analytique de $X$ d'int�rieur vide et $f$ est une fonction holomorphe sur $X \setminus H$ tel que, pour chaque $x \in H$, il existe un voisinage $U$ de $x$ dans $X$ et deux fonctions holomorphes $g$ et $h$ sur $V$ telles que les z�ros de $H$ soit inclus dans $H \cup V$ et le quotient $g/h$ co�ncide avec $f$ sur $V \setminus (V \cap H)$, pour la relation d'�quivalence d�finit comme suit : $(H,f)$ et $(H',f')$ sont dits �quivalents si $f$ et $f'$ co�ncident sur $M \setminus (H \cap H')$
\end{Def}
Une fonction holomorphe $g : X \to \C$ peut �tre vu comme une fonction m�romorphe donn�e par la classe d'�quivalence de $(\emptyset,g)$. R�ciproquement, on peut voir une fonction m�romorphe $(H,f)$ qui contient $(\emptyset,g)$ comme une fonction holomorphe. On dira que $g$ prolonge $f$ � $X$.

\begin{Thm}\label{caracnormal}
Un espace analytique complexe r�duit $X$ est normal si, et seulement si, pour tout ouvert $\Omega$ de $X$ et toute hypersurface $H \subset \Omega$ ferm�e et d'int�rieur vide dans $\Omega$, toute fonction holomorphe $g : \Omega \setminus H \to \C$ localement born�e le long de $H$ se prolonge en une fonction holomorphe sur $\Omega$.
\end{Thm}

\begin{Prop}\label{merom_local}
Soient $X$ un espace analytique complexe r�duit de dimension pure, $H$ un sous-ensemble analytique d'int�rieur vide de $M$ et $f : M \setminus H \to \C$ une fonction localement born�e le long de $H$. Alors $f$ est m�romorphe sur $X$.
\end{Prop}
\begin{Def}
Un espace complexe r�duit $M$ est dit faiblement normal si, pour toit ouvert $U$ de $M$, toute fonction continue sur $U$ est holomorphe sur un ouvert de Zariski dense de $U$ est holomorphe sur $U$.
\end{Def}

Cette propri�t� est plus faible que celle de normalit� en vertu de la caract�risation \ref{caracnormal} et de la proposition \ref{merom_local}

\begin{Thm}\label{norm_holom}
Soit $f : M \to N$ une application continue entre espaces analytiques complexes avec $M$ faiblement normal. Alors, les condition suivantes sont �quivalentes :
\begin{itemize}
	\item L'application $f$ est holomorphe.
	\item Le graphe de $f$ est un sous-ensemble analytique (ferm�) de $M \times N$
\end{itemize}
\end{Thm}
\begin{proof}
$ \Rightarrow $ : Le graphe de $f$ est le lieu des z�ros de $y-f(x)$ et est donc un sous-ensemble analytique de $M \times N$. \\
$ \Leftarrow $ : Soit $\pi : \Gamma_f \to M$ la projection du graphe de $f$ sur $M$. C'est un hom�omorphisme holomorphe ($\pi$ est donc propre). Il existe donc un ouvert de Zariski $U$ de $M$ tel que $\pi : \pi^{-1}(U) \to U$ soit une submersion et est donc un biholomorphisme ( par le th�or�me d'inversion locale). Ainsi, $\pi^{-1}$ est continue, et holomorphe sur $U$. Comme $M$ est faiblement normal alors $\pi$ est holomorphe sur $M$ et donc $f$ aussi. 
\end{proof}
\section{G�om�trie diff�rentielle}
\begin{Prop} 
Soient $X,Y$ deux vari�t�s de m�me dimension, $X$ est suppos�e compacte. Soient $f : X \to Y$ une application lisse et $y$ une valeur r�guli�re de $f$. Alors,
\begin{itemize}
	\item $f^{-1}(y)$ est fini
	\item Il existe un voisinage ouvert $V$ de $y$ tel que :
		\[ \forall z \in V, | f^{-1}(z)|=|f^{-1}(y)|
	\]
\end{itemize}
\end{Prop}
\begin{proof}
Lafontaine p 58
\end{proof}
\begin{Cor} \label{revet}
$f$ est un rev�tement sur l'ensemble des valeurs r�guli�res.
\end{Cor}

\begin{Thm}
Soit $X$ une vari�t� compacte connexe orient�e de dimension $n$. Une forme diff�rentielle de degr� $n$ est exacte si, et seulement si, elle est d'int�grale nulle
\end{Thm}

\begin{proof}
Lafontaine Thm 6 p 232
\end{proof}





\ifwhole
 \end{document}
\fi



\bibliographystyle{alpha} 
\bibliography{biblio2} 



 \end{document}
