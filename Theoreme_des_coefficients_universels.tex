\section{Th�or�me des coefficients universels}

\begin{Thm}[Th�or�me des coefficients universels]\label{homol_sing}
Soient $C=(C_\bullet,\partial)$ un complexe de chaines de $\Z$-modules libres, $G$ un groupe ab�lien. Alors, les groupes d'homologies $H_i(C,G)$ du complexe de chaines $C \otimes G=(C_\bullet \otimes G,\partial \otimes id)$ sont d�termin�s par les groupes d'homologie $H_i(C_\bullet)$ via la suite exacte courte :
	\[\xymatrix {
		0 \ar[r] & H_i(C,G) \ar[r]  & H_i(C \otimes G) \ar[r] & Tor(H_{i-1}(C),G) \ar[r] &0 \\ 
		}
\]
	o� $Tor$ est le foncteur d�riv� du produit tensoriel.
\end{Thm}
\begin{proof}
\cite{\bredon}
\end{proof}
Si $k$ est un corps de caract�ristique 0, $(k,+)$ est sans torsion et donc 
	\[\forall i \in \N, Tor(H_{i-1}(C),k)=0
\]
Ainsi, 
	\[ H_i(C) \otimes k=H_i(C \otimes k)
\]
Dans le cas de l'homologie singuli�re pour $X$ un espace topologique, 
\[ H_i(C,\Z) \otimes k=H_i(X,k)
\]
On en d�duit une inclusion de $H_i(C,\Z)$ modulo la torsion (que l'on notera $H_i(C,\Z)^{libre}$)  dans $H_i(X,\R)$. On en d�duit que $H_i(C,\Z)^{libre}$ forme un r�seau de $H_i(C,\R)$
